% !TeX root = Bericht_main.tex
\subsection{Analytische Betrachtung der linearen Transportgleichung}
\textbf{Bisher} wollten wir aus gegebenen Randwerten den Fluss $ q $ berechnen.\\
\textbf{Jetzt} ist $ \Omega \subseteq \R^2 $ (wir betrachten nur zweidimensionale Gebiete) der Fluss $ q: \Omega \to \R^2 $ mit $ \dive(q) = 0 $ gegeben. Gesucht wird nun die Dichteverteilung $ \rho: \Omega \times [0,T] \to \R_{\geq 0} $ einer 
transportierten Substanz. Gegeben ist dazu die \defi{Anfangsverteilung} der Substanz $ \rho_0:  \Omega \to \R_{\geq 0}$ und die \defi{Einfluss} der Substanz über Zeit $ \rhoin: \Gammain \times [0,T] \to \R_{\geq 0} $ mit $ \Gammain \coloneqq \{ z \in \partial \Omega: q(z) \cdot \nu(x) \leq 0 \} $ ($ \Gammain $ ist gerade die Menge der Randpunkte an dem Substanz einfließen kann).

Gegebenenfalls betrachten wir $ \rho $ auch als 
\[ \rho: [0,T] \to \Abb(\Omega, \R_{\geq 0}), t \mapsto \rho(\cdot,t). \]

Nach der physikalischen Modellierung gilt folgende Bilanzgleichung:
\begin{gather*}
	\forall \ K \subseteq \Omega, t \in [0,T]: \frac{\diff}{\diff t} \int_K \rho(x,t) \diff x + \oint_{\partial K} \rho(x,t) q(x) \cdot \nu(x) \diff a = 0 \\
	\stackrel{\text{Gauß}}{\implies} \int_K \partial_t \rho(x,t) + \dive(\rho q)(x,t) \diff x = 0 
\end{gather*}
Da $ K \subseteq \Omega $ belibeig ist und \unklar{$ \rho, q \in C^1(\Omega) $} folgt die \defi{lineare Transportgleichung}
\begin{gather*}
	\partial_t \rho + \dive(\rho q) = 0 \text{ in } \Omega \times (0,T) 
\end{gather*}
Mit der Randbedingung $ \rho(x,t) = \rhoin(x,t) \text{ auf } \Gammain \times (0,T)$ und der Anfangsbedingung $ \rho(x,0) = \rho_0(x) \text{ auf } \Omega $ folgt das zu lösende analytische Problem
\begin{gather*}
	\text{Bestimme } \rho: \Omega \times [0,T] \to \R_{\geq 0} \text{, sodass}\\
	\lTG
	\begin{cases}
		\partial_t \rho + \dive(\rho q) = 0 &\text{ in } \Omega \times (0,T)\\
		\rho(x,t) = \rhoin(x,t) &\text{ auf } \Gammain \times ()0,T)\\
		\rho(x,0) = \rho_0(x) &\text{ auf } \Omega
	\end{cases}
\end{gather*}



\subsection{Erhaltungsgrößen der analytischen Lösung}
\begin{Lemma}(Massenbilanz)
	\label{Massenbilanz}
	
	Sei $ \rho $ Lösung von $ \lTG $. 
	
	Dann gilt die \defi{Massenbilanz}:
	\[ \underbrace{\int_{\Omega} \rho(x,t) \diff x}_{(1)} = \underbrace{\int_{\Omega} \rho_0(x) \diff x}_{(2)}  \underbrace{-\int_{0}^{t} \int_{\Gammain} \rhoin(x,\tilde{t}) q(x) \cdot \nu(x) \diff a \diff \tilde{t}}_{(3)} \underbrace{- \int_{0}^{t} \int_{\Gammain} \rho(x,\tilde{t}) q(x) \cdot \nu(x) \diff a \diff \tilde{t}}_{(4)} \]
\end{Lemma}
	Interpretation der Terme:
	\begin{enumerate}[label=(\arabic*)]
		\item Masse zum Zeitpunkt $ t $,
		\item Masse zum Anfangszeitpunkt $ 0 $,
		\item Masse, die in $ [0,t] $ hinzu geflossen ist ($ \nu $ ist äußere Normale, deswegen \enquote{-} statt \enquote{+}),
		\item Masse, die in $ [0,t] $ abgeflossen ist.
	\end{enumerate}

\begin{Lemma}(Energiebilanz)
	\label{Energiegleichung}
	
	Sei $ \dive(q) = 0 $ und $ \rho $ eine Lösung von $ \lTG $. 
	
	Dann gilt die \defi{Energiebilanz}: $ \forall \ t\in [0,T] $
	\[\int_{\Omega} \abs{\rho(x,t)}^2 \diff x = \int_{\Omega} \abs{\rho_0(x)}^2 \diff x  + \int_{0}^{T} \left[ \int_{\Gammain}\abs{\rhoin(t)}^2 \abs{q\cdot n} \diff a - \int_{\Gammaout} \abs{\rho(t)}^2 \abs{q\cdot n} \diff a  \right] \]
\end{Lemma}

\subsection{Lösungsbegriffe von $ \lTG $}

\begin{define}(Lösungsbegriffe)
	
	\begin{itemize}
		\item Eine Lösung von
		\begin{gather*}
		\text{Bestimme } \rho: \Omega \times [0,T] \to \R_{\geq 0} \text{, sodass}\\
		\lTG
		\begin{cases}
		\partial_t \rho + \dive(\rho q) = 0 &\text{ in } \Omega \times (0,T)\\
		\rho(x,t) = \rhoin(x,t) &\text{ auf } \Gammain \times (0,T)\\
		\rho(x,0) = \rho_0(x) &\text{ auf } \Omega.
		\end{cases}
		\end{gather*}
		heißt \defi{klassische Lösung}.
%		\item Eine Lösung von 
%		\begin{gather*}
%		\text{Bestimme } \rho: \Omega \times [0,T] \to \R_{\geq 0} \text{, sodass}\\
%		\stlTG
%		\begin{cases}
%		\int_{\Omega} \int_{0}^{T} \partial_t \rho + \dive(\rho q) \phi \diff t \diff x = 0 &\text{ in } \Omega \times (0,T)\\
%		\rho(x,t) = \rhoin(x,t) &\text{ auf } \Gammain \times (0,T)\\
%		\rho(x,0) = \rho_0(x) &\text{ auf } \Omega
%		\end{cases}\\
%		\text{ für alle } \phi: \Omega \times (0,T) \to \R.
%		\end{gather*}
%		heißt \defi{starke Lösung}.
		\item Eine Lösung von 
		\begin{gather*} 
		\text{Bestimme } \rho \in L^1(\Omega \times [0,T], \R_{\geq 0}) \text{, sodass}\\
		\swlTG
		\begin{cases}
		\displaystyle
		\int_{\Omega} \rho_0 \phi(0) \diff x = \mkern-16mu &- \displaystyle \int_{0}^{T} \int_{\Omega} \rho (\partial_t \phi - q \nabla_{\unklar{x}} \phi ) \diff x \diff t \\
		&+\displaystyle\int_{0}^{T} \left[ \oint_{\Gammain} \rhoin q \cdot \nu \phi \diff a + \oint_{\Gammaout} \rho q \cdot n \phi \diff a \right] \diff t
		\end{cases}	\\
		\unklar{\text{ für alle } \phi: \Omega \times (0,T) \to \R \text{ mit } \phi(\cdot,T) = 0 \text{ auf } \Omega \text{ und } \phi = 0 \text{ auf } \Gammain \times (0,T).}
		\end{gather*}
		heißt \defi{schwache Lösung}.
	\end{itemize}
\end{define}

\begin{Lemma}(Zusammenhang der Lösungsbegriffe)
	
	\begin{enumerate}
		\item Ist $ \rho $ eine klassische Lösung, so ist $ \rho $ auch eine schwache Lösung.
		\item Ist $ \rho $ glatt genug und eine schwache Lösung, so ist $ \rho $ eine klassische Lösung. 
	\end{enumerate}
\end{Lemma}

\subsection{Numerische Approximation}
Ziel ist es nun eine \enquote{gute} Approximation für die schwache Lösung $ \rho $ von $ \swlTG $ zu finden. Dabei werden wir $ \rho $ zunächst im Raum diskretisieren um die semi-diskrete Lösung $ \rho_h $ zu erhalten. Anschließend diskretierien wir in der Zeit.

\subsubsection{Diskretisierung im Raum}
\unklar{Wir betrachten $ \rho $ als Lösung und schreiben das zu lösende Problem um.} Sei $ \mathcal{K} $ eine Triangulierung von $ \Omega $ wie bisher. 
\begin{define}(Flux)
	
	Zu gegebenen Fluss $ q:\Omega \to \R^2 $ definieren wir die \defi{Flussfunktion}/\defi{Flux} als
	\[ \Psi: \Abb(\Omega \times [0,T], \R) \to \Abb(\Omega \times [0,T], \R^2), \Psi(\rho) = \rho q \]
\end{define}

Dann gilt für eine klassische Lösung $ \rho $ von $ \lTG $  $ \partial_t \rho = - \dive(\Psi(\rho)) $ und somit
\begin{align*}
	&\sum_{K \in \mathcal{K}} \oint_{\partial K} \phi \Psi(\rho) \cdot n^K \diff a \stackrel{(\star)}{=} \oint_{\partial \Omega} \phi \Psi(\rho) \cdot n \diff a \stackrel{\text{Gauß}}{=} \int_{\Omega} \dive(\Psi(\rho) \Phi) \diff x \\
	& \quad = \int_{\Omega} \phi \dive(\Psi(\rho)) + \Psi(\rho) \cdot \nabla \phi \diff x = - \int_\Omega \partial_t \rho \phi \diff x + \int_\Omega \Psi(\rho) \cdot \nabla \phi \diff x \\
	\iff &\sum_{K \in \mathcal{K}} \left(\Psi(\rho) \cdot n^K, \phi \right)_{\partial K} = - \left(\partial_t \rho, \phi  \right)_\Omega - \left(\Psi(\rho), \nabla\phi  \right)_\Omega \\
	\iff & \left(\partial_t \rho, \phi  \right)_\Omega  = - \left(\Psi(\rho), \nabla\phi  \right)_\Omega - \sum_{K \in \mathcal{K} } \left(\Psi(\rho) \cdot n^K, \phi \right)_{\partial K} \\
	\iff & \left(\partial_t \rho, \phi  \right)_\Omega  = - \left(\Psi(\rho), \nabla \phi  \right)_\Omega - \sum_{K \in \mathcal{K}} \sum_{F \in \mathcal{F}_K} \left(\Psi(\rho) \cdot n^K, \phi \right)_{F} \\
	\iff & \sum_{K \in \mathcal{K}} \left(\partial_t \rho, \phi  \right)_K  = \sum_{K \in \mathcal{K}} \bigg( - \left(\Psi(\rho), \nabla \phi \right)_K\\
	&\qquad\qquad\qquad\qquad -\sum_{\substack{F \in \mathcal{F}_K \\ F \not\subseteq \Gammain}} \left(\Psi(\rho) \cdot n^K, \phi \right)_{F}- \sum_{\substack{F \in \mathcal{F}_K \\ F \subseteq \Gammain}} \left(\rhoin q \cdot n^K, \phi \right)_{F} \bigg)\\
	\iff & \sum_{K \in \mathcal{K}} \left(\partial_t \rho, \phi  \right)_K  = \sum_{K \in \mathcal{K}} \bigg( -\left(\Psi(\rho), \nabla\phi \right)_K \\
	&\qquad\qquad\qquad\qquad -\sum_{\substack{F \in \mathcal{F}_K \\ F \not\subseteq \Gammain}} \left(\Psi(\rho) \cdot n^K, \phi \right)_{F} - \left(\rhoin q \cdot n^K, \phi \right)_{\partial K \cap \Gammain} \bigg) \tag{\sun}
\end{align*}
\begin{remark}
	zu $ (\star) $: Integration über inneren Kanten $ F $ fällt weg, da für die beiden Nachbarzellen wegen der äußeren Normalen in unterschiedliche Richtungen integriert wird:
	\[ F = K \cap K': \oint_{F}  \phi \Psi(\rho) \cdot n^K \diff a = - \oint_{F} \phi \Psi(\rho) \cdot n^{K'} \diff a\]
\end{remark}
\ \newline
\textbf{Ziel}:
Wir wollen nun die semidiskrete Lösung $ \rho_h: [0,T] \to Q_h $ bzw eine Differenzialgleichung für $ \rho_h $ herzuleiten. 

\textbf{Approximation}:
Wähle Lösungs-/Ansatzraum $ Q_h = \Pi_{K\in\mathcal{K}} \mathbb{P}_k $ für $ k \in \mathbb{N}_0 $. Für $ k = 0 $ heißt $ Q_h $ \defi{Finite-Volumen-Raum}, für $ k \geq 1 $ nennt man $ Q_h $ \defi{Discontinuous-Galerkin-Raum}. Anders als bei den FE-Räumen wird für Elemente aus $ Q_h $ keine Stetigkeit auf $ \Omega $ gefordert. D.h. im Allgemeinen lässt sich $ u \in Q_h $ (Definiert auf $ \Omega_h $) nicht stetig auf $ \Omega $ fortsetzen, da für eine beliebige innere Kante $ \overline{F} = \partial K \cap \partial K' $ mit Grenzwert von $ K $ aus unterschiedlich zu dem von $ K' $ aus sein kann. Damit kann (\sun) nicht direkt auf $ Q_h $ übertragen werde, da noch zu klären ist, was $ \Psi(\rho) $ auf $ F \in \mathcal{F} $ bedeutet. Wir betrachten den \defi{numerischen Fluss} (numerical flux) $ \Psi^* $, welcher \enquote{in geeigneter Weiße} $ \Psi(\rho) $ auf den Kanten $ F \in \mathcal{F} $ ersetzt . Wir zählen $ \mathcal{K} $ durch mit $ \mathcal{K} = \{K_1 , \dots , K_N\} $, $ N \coloneqq \abs{\mathcal{K}} $.
Unser numerischer Ansatz wird durch (\sun) motiviert und lautet für alle $ \phi_h \in Q_h $
\[
\sum_{i = 1}^{N} \left(\partial_t \rho_h, \phi_h  \right)_{K_i}  = \sum_{i=1}^{N} \bigg(- \left(\Psi(\rho_h), \nabla\phi_h \right)_{K_i} -\sum_{\substack{F \in \mathcal{F}_{K_i} \\ F \not\subseteq \Gammain}} \left(\Psi^*(\rho_h) \cdot n^K, \phi \right)_{F} - \left(\rhoin q \cdot n^K, \phi \right)_{\partial K_i \cap \Gammain} \bigg)
\]



\textbf{Finite-Volumen}:
Zunächst betrachten wir den Finite-Volumen-Ansatzraum $ Q_h = \Pi_{K \in \mathcal{K}} \mathbb{P}_0 $ und als numerischen Fluss den \defi{upwind flux}. Dieser ist gegeben für alle $ K \in \mathcal{K} $ und $ F \in \mathcal{F}_K $ durch:
\begin{gather*}
	\text{Auf } F \text{ gilt }\Psi^*(\rho_h) = \begin{cases}
	\Psi(\rho_{h|K}) &\text{, für } q \cdot n^K > 0\\
	Psi(\rho_{h|K'}) &\text{, für } q\cdot n^K < 0 \text{ und } \overline{F} = \partial K \cap \partial K'\\
	0 &\text{, für } q\cdot n^K < 0 \text{ und } F \subseteq \Gammain.
	\end{cases}  
\end{gather*}


Weiter gilt für die Zellenbasis $ \{\eta_i\}_{i=1}^N $ mit $ \eta_i = \mathbbm{1}_{K_i} $ 
\begin{itemize}
	\item $ Q_h = \spann\{\eta_1, \dots , \eta_N\} $,
	\item $ \{\eta_i\}_{i=1}^N $ ist eine Orthogonalbasis (im Allegemine aber keine ONB da $ \norm{\eta_i}^2 = \lambda(K_i) $\,),
	\item $ \forall \ i \in \{1, \dots , N\}:  \supp(\eta_i) \subseteq \overline{K_i}$.
\end{itemize}



 
Durch einsetzen der Zeilenbasis und mit $ \supp(\eta_i) \subseteq \overline{K_i} (\ i \in \{1, \dots , N\})$ ergibt sich für alle $\ i \in \{1, \dots , N\}$ 
\[ \left(\partial_t \rho_h, \eta_i  \right)_{K_i}  = \bigg( -\underbrace{(\Psi(\rho_h), \nabla\eta_i)_{K_i}}_{=0 \text{, da } \nabla \eta_i = 0} - \sum_{\substack{F \in \mathcal{F}_{K_i} \\ F \not\subseteq \Gammain}} \left(\Psi^*(\rho_h) \cdot n^{K_i}, \eta_i \right)_{F} - \left(\rhoin q \cdot n^{K_i}, \eta_i \right)_{\partial K_i \cap \Gammain} \bigg)\]

\subsubsection{Diskretisierung in der Zeit}

 