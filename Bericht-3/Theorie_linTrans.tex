% !TeX root = Bericht_main.tex
\subsection{Analytische Betrachtung der linearen Transportgleichung}
\textbf{Bisher} wollten wir aus gegebenen Randwerten den Fluss $ q $ berechnen.\\
\textbf{Jetzt} ist $ \Omega \subseteq \R^D $ der Fluss $ q: \Omega \to \R^D $ mit $ \dive(q) = 0 $ gegeben. Gesucht wird nun die Dichteverteilung $ \rho: \Omega \times [0,T] \to \R_{\geq 0} $ einer 
transportierten Substanz. Gegeben ist dazu die \defi{Anfangsverteilung} der Substanz $ \rho_0:  \Omega \to \R_{\geq 0}$ und die \defi{Einfluss} der Substanz über Zeit $ \rhoin: \Gammain \times [0,T] \to \R_{\geq 0} $ mit $ \Gammain \coloneqq \{ z \in \partial \Omega: q(z) \cdot \nu(x) \leq 0 \} $ ($ \Gammain $ ist gerade die Menge der Randpunkte an dem Substanz einfließen kann).

Nach der physikalischen Modellierung gilt folgende Bilanzgleichung:
\begin{gather*}
	\forall \ K \subseteq \Omega, t \in [0,T]: \frac{\diff}{\diff t} \int_K \rho(x,t) \diff x + \oint_{\partial K} \rho(x,t) q(x) \cdot \nu(x) \diff a = 0 \\
	\stackrel{\text{Gauß}}{\implies} \int_K \partial_t \rho(x,t) + \dive(\rho q)(x,t) \diff x = 0 
\end{gather*}
Da $ K \subseteq \Omega $ belibeig ist und \unklar{$ \rho, q \in C^1(\Omega) $} folgt die \defi{lineare Transportgleichung}
\begin{gather*}
	\partial_t \rho + \dive(\rho q) = 0 \text{ in } \Omega \times (0,T) 
\end{gather*}
Mit der Randbedingung $ \rho(x,t) = \rhoin(x,t) \text{ auf } \Gammain \times (0,T)$ und der Anfangsbedingung $ \rho(x,0) = \rho_0(x) \text{ auf } \Omega $ folgt das zu lösende analytische Problem
\begin{gather*}
	\text{Bestimme } \rho: \Omega \times [0,T] \to \R_{\geq 0} \text{, sodass}\\
	\lTG
	\begin{cases}
		\partial_t \rho + \dive(\rho q) = 0 &\text{ in } \Omega \times (0,T)\\
		\rho(x,t) = \rhoin(x,t) &\text{ auf } \Gammain \times ()0,T)\\
		\rho(x,0) = \rho_0(x) &\text{ auf } \Omega
	\end{cases}
\end{gather*}

\subsection{Erhaltungsgrößen der analytischen Lösung}
\begin{Lemma}(Massenbilanz)
	\label{Massenbilanz}
	
	Sei $ \rho $ Lösung von $ \lTG $. 
	
	Dann gilt die \defi{Massenbilanz}:
	\[ \underbrace{\int_{\Omega} \rho(x,t) \diff x}_{(1)} = \underbrace{\int_{\Omega} \rho_0(x) \diff x}_{(2)}  \underbrace{-\int_{0}^{t} \int_{\Gammain} \rhoin(x,\tilde{t}) q(x) \cdot \nu(x) \diff a \diff \tilde{t}}_{(3)} \underbrace{- \int_{0}^{t} \int_{\Gammain} \rho(x,\tilde{t}) q(x) \cdot \nu(x) \diff a \diff \tilde{t}}_{(4)} \]
\end{Lemma}
	Interpretation der Terme:
	\begin{enumerate}[label=(\arabic*)]
		\item Masse zum Zeitpunkt $ t $,
		\item Masse zum Anfangszeitpunkt $ 0 $,
		\item Masse, die in $ [0,t] $ hinzu geflossen ist ($ \nu $ ist äußere Normale, deswegen \enquote{-} statt \enquote{+}),
		\item Masse, die in $ [0,t] $ abgeflossen ist.
	\end{enumerate}

\begin{Lemma}(Energiegleichung)
	\label{Energiegleichung}
	
	Sei $ \dive(q) = 0 $ und $ \rho $ eine Lösung von $ \lTG $. 
	
	Dann gilt die \defi{Energiebilanz}: $ \forall \ t\in [0,T] $
	\[\int_{\Omega} \abs{\rho(x,t)}^2 \diff x = \int_{\Omega} \abs{\rho_0(x)}^2 \diff x  + \int_{0}^{T} \left[ \int_{\Gammain}\abs{\rhoin(t)}^2 \abs{q\cdot n} \diff a - \int_{\Gammaout} \abs{\rho(t)}^2 \abs{q\cdot n} \diff a  \right] \]
\end{Lemma}

\subsection{Lösungsbegriffe von $ \lTG $}

\begin{define}(Lösungsbegriffe)
	
	\begin{itemize}
		\item Eine Lösung von
		\begin{gather*}
		\text{Bestimme } \rho: \Omega \times [0,T] \to \R_{\geq 0} \text{, sodass}\\
		\lTG
		\begin{cases}
		\partial_t \rho + \dive(\rho q) = 0 &\text{ in } \Omega \times (0,T)\\
		\rho(x,t) = \rhoin(x,t) &\text{ auf } \Gammain \times (0,T)\\
		\rho(x,0) = \rho_0(x) &\text{ auf } \Omega.
		\end{cases}
		\end{gather*}
		heißt \defi{klassische Lösung}.
		\item Eine Lösung von 
		\begin{gather*}
		\text{Bestimme } \rho: \Omega \times [0,T] \to \R_{\geq 0} \text{, sodass}\\
		\stlTG
		\begin{cases}
		\int_{\Omega} \int_{0}^{T} \partial_t \rho + \dive(\rho q) \phi \diff t \diff x = 0 &\text{ in } \Omega \times (0,T)\\
		\rho(x,t) = \rhoin(x,t) &\text{ auf } \Gammain \times (0,T)\\
		\rho(x,0) = \rho_0(x) &\text{ auf } \Omega
		\end{cases}\\
		\text{ für alle } \phi: \Omega \times (0,T) \to \R.
		\end{gather*}
		heißt \defi{starke Lösung}.
		\item Eine Lösung von 
		\begin{gather*}
		\text{Bestimme } \rho \in L^1(\Omega \times [0,T], \R_{\geq 0}) \text{, sodass}\\
		\swlTG
		\int_{\Omega} \rho_0 \phi(0) \diff x = \int_{0}^{T} \left[ - \int_{\Omega} \rho (\partial_t \phi - q \nabla_{\unklar{x}} \phi ) \diff x + \oint_{\Gammain} \rhoin q \cdot \nu \phi \diff a + \oint_{\Gammaout} \rho q \cdot n \phi \diff a \right] \diff t\\
		\unklar{\text{ für alle } \phi: \Omega \times (0,T) \to \R \text{ mit } \phi(\cdot,T) = 0 \text{ auf } \Omega \text{ und } \phi = 0 \text{ auf } \Gammain \times (0,T).}
		\end{gather*}
		heißt \defi{schwache Lösung}.
	\end{itemize}
\end{define}

\begin{Lemma}(Zusammenhang der Lösungsbegriffe)
	
	\begin{enumerate}
		\item Ist $ \rho $ eine klassische Lösung, so ist $ \rho $ auch eine schwache Lösung.
		\item Ist $ \rho $ glatt genug und eine schwache Lösung, so ist $ \rho $ eine klassische Lösung. 
	\end{enumerate}
\end{Lemma}



 