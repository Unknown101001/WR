% !TeX root = Bericht_main.tex
\subsection{Die Konvektions-Reaktions-Diffusions-Gleichung}
Wir wollen, im Vergleich zum vorherigen Kapitel, neben dem Transport einer Substanz nun auch Diffusion und chemische Reaktionen der Substanz simulieren. Dazu seien die folgenden Größen vorgegeben:
\begin{align*}
&\text{\underline{\textbf{Geg.:}}}\\
&&&&&\Omega \subseteq \R^d \text{ Gebiet}\\
&&\text{\defi{Diffusionstensor}} &&&\kappa_c:\Omega \to \R^{d \times d}_{\sym}\text{ mit } \kappa_c^T \equiv \kappa_c \text{ und }\\
								&&&&& \qquad \exists \ \lambda_c > 0 \ \forall \ y \in \R^d: y^T \kappa_c y \geq \lambda_c \abs{y}^2\\
&&\text{\defi{Flussvektor}} &&&q:\Omega \to \R^d \text{ mit } \dive(q) = 0\\
&&\text{\defi{Reaktionsrate}} &&&r:\Omega\times[0,T]\times\R \to \R\\
&\text{\underline{\textbf{Ges.:}}}\\
	&&\text{\defi{Konzentrationsrate der Stoffes}} &&&c:[0,T]\times\Omega \to \R
\end{align*}
Aus der Physik erhalten wir wieder Bilanz- und Konstitutionsgleichungen:\\
\underline{\textbf{Bilanzgleichung:}}
\begin{gather*}
\forall \ (t_0,t_1)\times K \subseteq [0,T] \times \Omega: \\
\underbrace{\int_K \left( c(t_1,x) - c(t_0,x) \right) \diff x}_{\text{Konzentrationsänderung auf } K\text{ in } (t_0,t_1)} + \underbrace{\int_{t_0}^{t_1} \oint_{\partial K} \Psi(c) \cdot \nu \diff a \diff t }_{\text{Konzentrationsänderung auf dem Rand } \partial K \text{ in } (t_0,t_1) } \\
= \underbrace{\int_{t_0}^{t_1} \int_K r(t,x,c(t,x)) \diff x \diff t}_{\text{Reaktion (Produktion und Abbau) auf } K \text{ in } (t_0,t_1) }
\end{gather*}
\underline{\textbf{Konstitutionsgl.:}}
\begin{gather*}
	\Psi(c) = \underbrace{c \cdot q}_{\text{vgl. \enquote{Transportgl}}} - \underbrace{\kappa_c \nabla c}_{vgl \enquote{Potentialströme}}
\end{gather*}
Es folgt, die als \defi{Konvektions-Reaktions-Diffusions-Gleichung} bezeichnete Gleichung
\[ \partial_t c + \dive(c q - \kappa_c \nabla c) = r(c) \text{ in } (0,T) \times \Omega \]
Wie sonst geben wir uns auch Anfangs- und Randwerte in Form von $ c_0, c_D, g_N, g_R $vor und erhalten das Problem
\begin{gather*}
	\text{Bestimmme } c:[0,T] \times \Omega \to \R \text{, sodass gilt}\\
	\KDR \begin{cases}
		\text{\textbf{KDR-Gleichung: }}&\partial_t c + \dive(c q - \kappa_c \nabla c) = r(c) \text{ in } (0,T) \times \Omega\\
		\text{\textbf{Anfangswert: }} &c(0,x) = c_0(x) \text{ in } \Omega\\
		\text{\textbf{Randwert:}}\\
		\qquad \text{Dirichlet: } &c(t,x) = c_D(t,x) \text{ auf } (0,T) \times \GammaD \\
		\qquad \text{Neumann: } &-\kappa_c \nabla c(t,x) \cdot \nu = g_N(t,x) \text{ auf } (0,T) \times \GammaN \\
		\qquad \text{Robin: } &-\kappa_c \nabla c(t,x) \cdot \nu + \alpha_R c(t,x) = g_R(t,x) \text{ auf } (0,T) \times \GammaR
	\end{cases}\\
	\text{wobei } \Omega = \GammaD \cupdot \GammaN \cupdot \GammaR
\end{gather*}

\begin{remark}
	(Modellierung mit $ r $)
	\begin{itemize}
		\item $ r $ unabhängig von $ c $, also $ r(t,x,c) = r(t,x) \begin{cases}
		> 0, &\text{Quelle}\\
		< 0, &\text{Senke}
		\end{cases} $
		\item $ r(t,x,c) = r_0 c $ für $ r_0 = \text{const.} \begin{cases}
		> 0, &\text{exp. Wachstum}\\
		< 0, &\text{exp. Zerfall}
		\end{cases} $
	\end{itemize}
\end{remark}

\begin{remark}
	(Durch Modell abgedeckte Teilprobleme)
	\begin{align*}
	&\text{Transport/Konvektion} & \partial_t c + \dive(cq) &= 0\\
	&\text{Reaktion} & \partial_t c &= r(c)\\
	&\text{Diffusion} & \partial_t c &= \dive(\kappa_c \nabla c)
	\end{align*}
\end{remark}

\begin{Lemma} \label{schwache Formulierung zu KDR}
	(Schwache Formulierung zu $ \KDR $)\\
	Sei c eine Lösung vom Problem $ \KDR $ (d.h. $ c(0,\cdot) \equiv c_0 \text{ in } \Omega $ und $ c_{|\GammaD} \equiv c_D \text{ in } (0,T)~\times~\GammaD $).
	
	Dann gilt $ \forall t \in (0,T) $
	\begin{gather*}
		\int_\Omega \left( \partial_t c \, \phi + \kappa_c \nabla c \cdot \nabla \phi + \nabla c \cdot q \phi \right) \diff x + \oint_{\GammaR} \alpha_R \, c \, \phi \diff a\\ = \int_\Omega r(c) \phi \diff x + \oint_{\GammaN} g_N \phi \diff a + \oint_{\GammaR} g_R \phi \diff a
	\end{gather*}
	für alle Testfunktionen $ \phi $  mit $ \phi_{|\GammaD} = 0 $.
\end{Lemma}

Wir schreiben das Problem $ \KDR $, genauer die schwache Formulierung von $ \KDR $ aus Lemma \ref{schwache Formulierung zu KDR}, mithilfe der folgenden Operatoren um
\begin{define}(Operatoren für KDR)
	\begin{itemize}
		\item Definiere A durch
		\[ (A\, c , \phi)_\Omega = \int_\Omega (\kappa_c \nabla c \cdot \nabla \phi + \nabla c \cdot q \, \phi ) \diff x + \oint_{\GammaR} \alpha_R \, c \, \phi \diff a  \]
		\item Definiere M durch
		\[ (M \, c, \phi)_\Omega = \int_\Omega c \, \phi \diff x \]
		\item Definiere R durch
		\[ (R\, c, \phi)_\Omega = \int_\Omega r(c) \, \phi \diff x + \oint_{\GammaN} g_N \, \phi \diff a + \oint_{\GammaR} g_R \, \phi \diff a \]
	\end{itemize}
	für alle Testfunktionen $ \phi $ mit $ \phi_{|\GammaD} = 0 $.
\end{define}

Wir erhalten somit die alternative Formulierung:
\begin{gather*}
	\text{Bestimme } c:[0,T] \times \Omega \to \R \text{, sodass gilt}\\
	\OpKDR\begin{cases}
		(M \partial_t c, \phi)_\Omega + (A c, \phi)_\Omega = (R c, \phi)_\Omega 
	\end{cases}\\
	\text{für alle Testfunktionen } \phi \text{ mit } \phi_{|\GammaD} = 0.
\end{gather*}

\subsection{Diskretisierung der KDR-Gleichung}
Erneut wählen $ V_h \coloneqq \spann_{z \in \mathcal{V}_h} \{ \lambda_z \} $ also Spann der Eckenbasis/\enquote{Hütchenfunktionen} und definieren
\begin{itemize}
	\item $ A_h: V_h \to V_h $ gegeben durch $ (A_h \, c_h, \phi_h)_\Omega = (A \, c_h, \phi_h) \ (c_h, \phi_h \in V_h)$
	\item $ M_h: V_h \to V_h $ gegeben durch $ (M_h \, c_h, \phi_h)_\Omega = (M \, c_h, \phi_h) \ (c_h, \phi_h \in V_h)$
	\item $ R_h(c): V_h \to V_h $ gegeben durch $ (R_h(c_h), \phi_h)_\Omega = (R(c_h), \phi_h) \ (c_h, \phi_h \in V_h)$
\end{itemize}
und die dazugehörigen Abbildungsmatrizen
\begin{itemize}
	\item $\underline{A} \coloneqq \bigg( (A_h \lambda_k, \lambda_j) \bigg)_{k,j} \in \R^{N \times N}$
	\item $\underline{M} \coloneqq \bigg( (M_h \lambda_k, \lambda_j) \bigg)_{k,j} \in \R^{N \times N}$
	\item $ \underline{R}(\underline{c}) \coloneqq \bigg( (R(c_h),\lambda_k) \bigg)_k \in \R^N \text{ wobei } c_h = \sum_k \underline{c}[k] \, \lambda_k $
\end{itemize}
Die Diskretisierung erfolgt wieder zunächst im Raum und dann in der Zeit:

Zunächst die semi-diskrete \defi{Galerkin-Approximation} (für festes $ t \in [0,T]$) 
\begin{gather*}
	\text{Bestimme } c_h(t,\cdot) \in V_h(c_D(t)) \text{, sodass gilt}\\
	\sdKDR\begin{cases}
		(M_h \, \partial_t c_h + A_h \, c_h, \phi_h)_\Omega = (R_h(c_h), \phi_h)_\Omega
	\end{cases}\\
	\text{ für alle Testfunktionen} \phi_h \in V_h(0) 
\end{gather*}

beziehungsweise in Matrizenschreibweise 
\[ \underline{M} \, \underline{\dot{c}} + \underline{A} \, \underline{c} = \underline{R}(\underline{c})  \]

als ODE in $ \R^N $.\\
Nun diskretisieren wir mit dem impl. Euler-Verfahren auch in der Zeit und erhalten für $ t_n \coloneqq n \, \Delta t  $ das voll-diskrete Problem
\begin{gather*}
\vdKDR \begin{cases}
	\left( \frac{1}{\Delta t} M_h(c_h^n - c_h^{n-1}) + A_h c_h^n, \phi_h \right)_\Omega = (R_h^n(c_h^n),\phi_h)_\Omega \quad (\phi_h \in V_h(0))
\end{cases}
\end{gather*}

\begin{define}(Operator für $ \vdKDR $)
	\begin{align*}
		&F_h^n: V_h \to V_h \text{ gegeben durch }\\ & \qquad (F_h^n(c_h), \phi_h) = \bigg( \frac{1}{\Delta t} M_h \, c_h + A_h \, c_h, \phi_h  \bigg)_\Omega + \bigg( R_h^n(c_h), \phi_h \bigg)_\Omega - \bigg( \frac{1}{\Delta} M_h c_h^{\textcolor{red}{n-1}}, \phi_h \bigg)_\Omega \\
		&\text{und } J_h^n \coloneqq D F_h^n
	\end{align*} 
\end{define}

\begin{Satz}(Lösbarkeit von $ \vdKDR $ und Konvergenz des Verfahrens)\\
	Sei $ \kappa_c $ uniform positiv definit, \unklar{$ r = \frac{1}{\Delta t} - R' $ }monoton fallend in der dritten Komponente (also $ \partial_3 r \le 0 $), weiter gelte $ q \cdot n \ge 0 \text{ auf } \GammaN $ und $ \alpha_R + \frac{1}{2} q \cdot n \ge 0 \text{ auf } \GammaR $.
	
	Dann hat $ \vdKDR $ eine eindeutige Lösung und die Jacobi-Matrix $ J_h^n $ ist positiv definit.
	
\end{Satz}

\begin{remark}\
	\begin{enumerate}
		\item Die ODE $\underline{M} \, \underline{\dot{c}} + \underline{A} \, \underline{c} = \underline{R}(\underline{c})$ ist dissipativ, daher sind B-stabile Runge-Kutter-Verfahren geeignet
		\item Für $ \partial_3 r > 0 $ und $ R' $ beschränkt, konvergiert das \unklar{Newton}-Verfahren mit $ \frac{1}{\Delta t} - \partial_3 r >0$ $ \implies$ Zeitschrittbeschränkung
	\end{enumerate}
\end{remark}

\subsection{Analyse für verschwindende Diffusion: $ \kappa_c \to 0 $}
\label{Analyse für verschwindende Diffusion}
Wir betrachten die verschwindende Diffusion ($ \kappa_c \ to 0 $) nur im Spezialfall
\begin{itemize}
	\item  $ \exists \kappa_0 > 0 (const.): \kappa_c = \kappa_0 \, I_n $ (stationärer Diffusion)
	\item $ \GammaR = \Gammaout $
	\item $ \alpha_R = q \cdot n $
	\item $ g_R = 0 $
	\item $ r = 0 $
\end{itemize}
und erhalten somit das Problem
\begin{gather*}
	\text{Bestimme }  c:[0,T] \times \Omega \to \R \text{, sodass gilt}\\
	\sDKDR\begin{cases}
		\partial_t c + \dive(c q - \kappa_0 \nabla c) = 0 &\text{, in } (0,T) \time \Omega\\
		c(0) = c_0  &\text{, in } \Omega\\
		c = c_D &\text{, auf } (0,T) \times \GammaD \\
		\kappa_0 \nabla c \cdot n + (q \cdot n) c = 0 &\text{, auf } (0,T) \times \GammaR
	\end{cases}
\end{gather*}

\begin{examples}(in 1D)
	\label{1D-Bsp}

	Mit $ \Omega = (0,1), q \equiv 1, \kappa_0 > 0, \GammaD = \{0\} \text{ und } \GammaR = \{1\}$. Außerdem sei $ \kappa_0 $ sehr klein ($ \kappa_0 << 1 $).
	
	Es gilt für $ \sDKDR $
	\begin{gather*}
	\text{Bestimme }  c:[0,T] \times (0,1) \to \R \text{, sodass gilt}\\
	\begin{cases}
		-\kappa_0 \, c'' + c' = 0 \\
		c(0) = 1  \\
		\kappa_0 c'(1) + c(1) = 0
	\end{cases}
	\end{gather*}
	
	und für die lineare Transportgleichung $ \lTG $
	\begin{gather}
		\text{Bestimme } \rho:[0,T] \times (0,1) \to \R \text{, sodass gilt}\\
		\begin{cases}
			\rho' = 0\\
			\rho(0) = 1
		\end{cases}
	\end{gather}
	
	Wir erhalten die Lösungen
	\begin{align*}
		&c(x) = a_1 + a_2 \exp(\frac{x}{\kappa_0}) \text{ mit } a_1 = \frac{2}{2 - \exp(- \frac{1}{\kappa_0})} \text{ und } a_2 = \frac{1}{1 - 2 \exp(\frac{1}{\kappa_0})}\\
		&\rho(x) = 1
	\end{align*}
\end{examples}

\begin{Lemma}(A-Priori-Schranke)
	
	Es gilt
	\begin{align*}
		\int_\Omega \abs{c(T)}^2 \diff x \leq \int_\Omega \abs{c_0}^2 + \int_{0}^{T} \left( \oint_{\GammaD} c_D^2 \abs{q \cdot n} \diff a + 2 \oint_{\GammaD} \kappa_0 \nabla c \cdot n \, c_D \diff a\right) \diff t
	\end{align*}
\end{Lemma}

\begin{Lemma}(Konvergenz für $ \kappa_0 \to 0 $)
	
	Sei $ c $ eine Lösung von $ \sDKDR $ und $ \rho $ eine Lösung von $ \lTG $, also 
	\[ \partial_t \rho + \dive(\rho q) = 0, \rho = \rho_0 = c_D \text{ auf } \GammaD \text{ und }  \rho(0) = \rho_0 = c_0.  \]
	
	Dann gilt 
	\[ \int_\Omega \abs{c(T) - \rho(T)}^2 \diff x \leq \kappa_0 \int_{0}^{T} \norm{\nabla c}_\Omega\, \norm{\nabla(\rho - c)}_\Omega \diff t \]
\end{Lemma}

\begin{Korollar}~
	
	Falls $ \norm{\nabla c}_\Omega $ und $ \norm{\nabla \rho}_\Omega $ unabhängig von $ \kappa_0 $ beschränkt sind, folgt
	\[ c \stackrel{\mathcal{L}^2}{\to} \rho \ (\kappa_0 \to 0) \]
\end{Korollar}

\subsection{Streamline-Upwind-Petrov-Galerkin}
In diesem Abschnitt wollen wir eine Lösungsmethode für sehr kleine Diffusion finden. Wir betrachten weiterhin nur stationäre Diffusion (d.h. $ \exists \kappa_0 > 0: \kappa_c \equiv \kappa_0 I_n $) und Linearisieren die Reaktion (d.h. $ R(c) \approx r c \text{ mit } \unklar{r =  \frac{1}{\Delta t} - R'}$)

Im vorherigen Abschnitt \ref{Analyse für verschwindende Diffusion} haben wir bereits gesehen, dass unter gewissen Voraussetzungen die Lösung von $ \sDKDR $ gegen die Lösung der linearen Transportgleichung $ \lTG $ konvergiert (für $ \kappa_0 \to 0 $). Daher liegt es nahe für kleine $ \kappa_0  $ ähnliche Ideen zu verwenden (auch wenn durch $ \kappa_0 \neq 0 $ die Ideen nicht direkt übertragen werden können).

Wir haben bei der \enquote{Finite-Volumen} bzw. \enquote{dG}-Methode gesehen, das die Wahl von unstetigen (somit inbs. nicht-konformen) Approximationsräumen durchaus sinnvoll sein kann. Jedoch werden wir im folgenden nur die Testfunktionen unstetig und den Lösungsraum konform wählen.

Wir betrachten nun folgendes Problem:
\begin{gather*}
	\text{Bestimme } c:[0,T] \times \Omega \to \R \text{, sodass gilt}\\
	\begin{cases}
		- \dive(\kappa_0 \nabla c) + q \cdot \nabla c + r \, c = f &\text{, in } \Omega\\
		c = c_D &\text{, auf } \GammaD (\text{wir fordern } \Gammain \subseteq \GammaD)\\
		\kappa_0 \nabla c \cdot n = g_N &\text{, auf } \GammaN\\
		\kappa_0 \nabla c \cdot n  + \underbrace{\alpha_R}_{= q \cdot n} c = g_R &\text{, auf } \GammaR
	\end{cases}
\end{gather*}
Insbesondere betrachten wir $ \kappa_0 << 1 $ und $ \norm{q}_\infty \approx 1 $

\begin{examples}(In der Situation von Beispiel \ref{1D-Bsp})
	
	Wir betrachten wieder $ \Omega = (0,1) $ mit der äquidistanten \enquote{Triangulierung}
	 \[ \mathcal{K} \coloneqq \left\{K_j\mid j \in \{1, \dots ,N\}\right\} \text{ wobei } K_j \coloneqq (x_{j-1}, x_j), x_j \coloneqq j \cdot h \text{ und } h \coloneqq \frac{1}{N}.\]
	 
	 \begin{itemize}
	 	\item \underline{Finite Elemente,$ \kappa_0 << h $ und \enquote{central flux}:}
	 	
	 	Weiter arbeiten wir mit der Eckenbasis/\enquote{Hütchenfunktionen}:
	 	\[ \lambda_j(x) \coloneqq \max\left\{ 0, 1 - \frac{\abs{x-x_j}}{h} \right\} \quad (j \in [N])\]
	 	und 
	 	\begin{align*}
		 	A[j,k] \coloneqq& \int_{0}^{1} (\kappa_0 \lambda_k' \lambda_j' + \lambda_k \lambda_j) \diff x\\
		 	=& \begin{dcases}
			 	\frac{\kappa_0}{h^2}2h &\text{, für } k = j\\
			 	-\frac{\kappa_0}{h^2} h \pm \frac{1}{2h} h &\text{, für } k = j \pm 1 \\
			 	0 &\text{, sonst}
		 	\end{dcases} \quad (j,k\in[N])
	 	\end{align*}
	 	
	 	Somit ergibt sich
	 	\[\kappa_0 << 1 \implies A \approx \frac{1}{2} \begin{pmatrix}
	 	0& 1& & \\
	 	-1& \ddots&\ddots & \\
	 	& \ddots& \ddots& 1\\
	 	& & -1& 0\\
	 	\end{pmatrix} \]
	 	
	 	\item \underline{Finite Volumen. $ \kappa_0 = 0 $ und \enquote{upwind flux}:}
	 	
	 	Es kommt raus:
	 	\[A_0 =  \begin{pmatrix}
	 	1& & & \\
	 	-1& \ddots& & \\
	 	& \ddots& \ddots& \\
	 	& & -1& 1\\
	 	\end{pmatrix}  \]
	 \end{itemize}
\end{examples}

\underline{\textbf{Beobachtung:}}
\begin{enumerate}
	\item Die Lösung mit \enquote{central flux} kann stark oszillieren.
	\item Die Lösung mit \enquote{uqwind} ist monoton, die Matrix $ A_0 $ ist eine M-Matrix, also $ A_0^{-1}[j,k]~\geq~0 $ für alle $ j,k $.
	\item Die Differenz
	\[ A_0 - A \approx \frac{1}{2}
	\begin{pmatrix} 
	2 & -1 & & 0\\
	-1& \ddots &\ddots & \\
	  & \ddots& \ddots & -1\\	
	0 & & -1 & 2\\		
	\end{pmatrix} \] 
	entspricht einer Diffusionsmatrix $ \approx> $ \enquote{upwind} = \enquote{central flux} + \enquote{diffusion}
\end{enumerate}

\subsubsection{Variationsformulierung}
\begin{gather*}
	\text{Bestimmme } c:[0,T] \times \Omega \to \R \text{, sodass gilt}\\
	\begin{cases}
		c_{|\GammaD} = c_D &\\
		\mathcal{A}(c,\phi) = b(\phi) &(\phi: \overline{\Omega} \to \R, \phi_{|\GammaD} = 0)
	\end{cases}
\end{gather*}
\begin{align*}
	\text{wobei } &\mathcal{A}(c,\phi) = \int_\Omega (\kappa_0 \nabla c \cdot \nabla \phi + q \cdot \nabla c \phi + r \, c \, \phi) \diff x + \oint_{\GammaR} \alpha_R \phi \diff a\\
	\text{und } &b(\phi) = \oint_{\GammaN} g_N \phi \diff a + \oint_{\GammaR} g_R \phi \diff a + \int_\Omega f \phi \diff x.
\end{align*}

\underline{\textbf{Wahl der Lösungs- /Testräume:}}
\begin{itemize}
	\item \defi{Galerkin}: Lösungs- und Testraum gleich\\ $ c_h, \phi_h \in V_h$
	\item \defi{Petrov-Galerkin}: Lösungsraum und Testraum unterschiedlich\\ Lösungsfunktion $ c_h \in V_h $ und Testfunktion $ \phi_h + \delta_K q \cdot \nabla \phi_h $ auf $ K \in \mathcal{K} $ und mit $ \delta_K > 0 $.
\end{itemize}

\begin{define}(Approximierte (Bi-)Linearform)
	\begin{align*}
		\mathcal{A}_h(c_h, \phi_h) &\coloneqq \mathcal{A}(c_h, \phi_h) + \sum_{K \in \mathcal{K}} \delta_K + \left(-\dive(\kappa_0 \nabla c_h) + q \nabla c_h + r c_h, q \cdot \nabla \phi_h\right)_K \\
		b_h(\phi_h) &\coloneqq b(\phi_h) + \sum_{K \in \mathcal{K}} \delta_K \left( f, q\cdot \nabla \phi_h \right)_K
	\end{align*}
\end{define}

\begin{define}(\defi{P\'{e}clet-Zahl})
	\[ \Pe_K \coloneqq \frac{h \, \norm{q}_{\infty, K}}{\norm{\kappa}_{\infty, K}} \]
\end{define}

\begin{define}(\defi{Streamline-Diffusion-Norm})
	 \[ \norm{\phi_h}_{\SD} \coloneqq \left( \frac{\kappa_0}{2} \norm{\nabla \phi_h}_\Omega^2 + \frac{\unklar{r_0}}{2} \norm{\phi_h}_\Omega^2+ \sum_{K \in \mathcal{K}} \frac{\delta_K}{2} \norm{q \cdot \nabla \phi}_K^2  \right)^{1/2}  \]
\end{define}

\begin{Lemma}(Koerzivität von $ \mathcal{A}_h $)
	
	Sei $ \kappa_c = \kappa_0 I_n, \kappa_0 > 0, r \geq r_0 > 0, \Gammain \subseteq \GammaD, q \text{ mit } \dive(q) = 0 $ und \[ 0 < \delta_K \leq \frac{1}{2} \min\left\{ \frac{\kappa_0 h_K^2}{\norm{\kappa_0}_{\infty,K}^2 \CinV^2} , \frac{r_0}{\norm{r}_{\infty, K}}   \right\}\] mit $ \CinV > 0 $, sodass  \[ \norm{\dive(\kappa_0 \nabla \phi)}_K \leq h_K^{-1} \norm{\kappa}_{\infty,K} \CinV \norm{\nabla \phi}_K. \]
	
	Dann gilt für alle $ \phi_h \in V(0) $
	\[ \mathcal{A}_h(\phi_h, \phi_h) \geq \norm{\phi_h}_{\SD}^2. \]
\end{Lemma}

\begin{remark}\
	\begin{enumerate}[label=\Alph*)]
		\item $ \CinV $ hängt \textbf{nur} vom Gitter ab und ist bei uniformer Verfeinerung beschränkt.
		\item Für $ \Pe_K < 1 $ ist $ \delta_K \in \mathcal{O}(h_K) $.
		\item \textbf{Anwendung}: Wenn $ c_D $ nach $ \overline{\Omega} $ fortsetzbar ist, so ist
		\begin{gather*}
			\begin{rcases}
				\mathcal{A}_h(\underbrace{c_h-c_D}_{\in V_h(0)}, c_h - c_D ) \geq \norm{c_h - c_D}_{\SD}^2 \\
				\abs{b_h(c_h-c_D)} \leq C(f,g_R, g_N) \norm{c_h - c_D}_{\SD}
			\end{rcases} \implies \norm{c_h}_{\SD} \leq \norm{c_D}_{\SD} + C(f,g_R,g_N)\\
			\implies A \coloneqq \left( \mathcal{A}_h(\lambda_j, \lambda_k) \right)_{j,k} \in \R{N \times N} \text{ ist positiv definit} \approx> \text{ Stabilität für } \kappa_0 \to 0.
		\end{gather*}
	\end{enumerate}
\end{remark}
