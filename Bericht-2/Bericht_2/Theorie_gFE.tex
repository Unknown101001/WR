% !TeX root = Bericht_main.tex

\subsection{Analytische Herleitung}

\textbf{Ziel:} Wir wollen das Problem nun in einer Weise diskretisieren, sodass die Flussbilanz automatisch erfüllt ist (somit Flux Loss = 0).

Die Flussbilanz ist für die numerische Lösung erfüllt, wenn gilt:
\[ \int_{\Gamma_\text{in}} q_h \cdot n \diff a = - \int_{\Gamma_\text{out}} q_h \cdot n \diff a \]
Bezeichnungen:
\begin{itemize}
	\item $ \Gamma_\text{in} $ und $ \Gamma_\text{out} $ analog definiert zu Lemma (1.1) im Skript
	\item n als äußerer Normalenvektor
	\item (Bisher bei Finiten Elementen) $ q_h \coloneqq \kappa \nabla_h u_h$ wobei $ \nabla_h $ eine \enquote{passende} numerische Differentiation-Methode ist.
\end{itemize}

Um die Flussbilanz zu erfüllen, verwenden wir \defi{gemischte Finite Elemente}. Bei dieser Methode wird, im Gegensatz zu (linearen) Finiten Elementen, nicht nur $ u_h $ als numerische Approximation für $ u $ berechnet, sondern das Paar $ (u_h,q_h) $ als numerische Approximation an $ (u,q) $. Wie genau dies funktioniert, untersuchen wir im Folgenden:

Betrachten wir noch einmal unser eigentliches analytisches Problem (LG: Laplace-Gleichung) 
\begin{gather*}
 \text{Bestimme } u:\overline{\Omega} \to \R \text{ mit }\\	
	\text{(LG)}\begin{cases}
	\dive(- \kappa \nabla u )&= 0 \text{ in } \Omega\\
	u &= u_D \text{ auf } \Gamma_D \\
	(\kappa\nabla u) \cdot n &= g_N \text{ auf } \Gamma_N \\
	\end{cases}
\end{gather*}

und $ \kappa $ ist sym. pos. def. .
%und setzen den \defi{Fluss} zu 
%\[ q \coloneqq - \kappa \nabla u. \]
Dann können wir unser analytisches Problem umschreiben zu 
\begin{gather*}
	\text{Bestimme } u:\overline{\Omega} \to \R \text{ und } q: \overline{\Omega} \to \R^2 \text{ mit } \\
		\text{(LG F)}\begin{cases}
		\dive q &= 0 \text{ in } \Omega\\
		- \kappa \nabla u &= q \text{ in } \Omega\\
		u &= u_D \text{ auf } \Gamma_D \\
		-q \cdot n &= g_N \text{ auf } \Gamma_N \\
		\end{cases}
\end{gather*} 
der \defi{Laplace-Gleichung mit Fluss}. Nun schauen wir uns, wie zuvor bei Finiten Elementen, die schwache Formulierung von (LG F) an.
\begin{align*}
	\int_\Omega \dive(q)\, \Phi \diff x = 0 &\text{ für alle Testfunktionen } \Phi: \Omega \to \R \\
	\underbrace{\int_{\Omega} \kappa^{-1} (q + \kappa \nabla u) \cdot \Psi \diff x = 0}_{\int_{\Omega} \nabla u \cdot \Psi \diff x = - \int_{\Omega} (\kappa^{-1} q) \cdot \Psi \diff x}  &\text{ für alle Testfunktionen } \Psi:\Omega \to \R^2 \ (\star)
\end{align*}
Weiter wir wollen nun noch die Dirichlet-Randbedingung $ 	u = u_D \text{ auf } \Gamma_D $ in die schwache Formulierung einfließen lassen. Dazu verwenden wir den Satz von Gauß:
\[ \int_{\partial\Omega} (u\Psi) \cdot n \diff a \stackrel{\text{Gauß}}{=} \int_{\Omega} \dive(u\Psi) \diff x = \int_{\Omega} \nabla u \cdot \Psi \diff x + \int_{\Omega} u \dive(\Psi) \diff x \quad (\Psi:\Omega \to \R^2) \]
Wählen wir nun unseren Ansatzraum so, dass  für die Funktion $ \Psi: \Omega \to \R$ gilt $ \Psi \cdot n = 0 \text{ auf } \Gamma_N $. Damit folgt
\begin{align*}
	\int_{\Gamma_D} (u_D\Psi) \cdot n \diff a \overset{\Psi \cdot n|_{\Gamma_N} = 0}{\underset{u |_{\Gamma_D} = u_D} {=}} \int_{\partial\Omega} (u\Psi) \cdot n \diff a = \underbrace{\int_{\Omega} \nabla u \cdot \Psi \diff x}_{\stackrel{(\star)}{=}- \int_{\Omega} (\kappa^{-1} q) \cdot \Psi \diff x } + \int_{\Omega} u \dive(\Psi) \diff x.
\end{align*}
Die Neumann-Randbedingung $ (\kappa\nabla u) \cdot n = g_N \text{ auf } \Gamma_N $ wird durch die Wahl des Lösungsraumes erfüllt.


Wir erhalten die gewünschte schwache Formulierung, im Folgenden \defi{gemischtes Finite-Elemente-Problem} genannt:
\begin{gather*}
	\text{Bestimme } (q,u) \text{ mit } q\cdot n = -g_N \text{ auf } \Gamma_N \text{ und}\\
	\text{(gFE)}\begin{cases}
		\int_{\Omega} \kappa^{-1} q \cdot \Psi \diff x \, - \mkern-15mu &\int_{\Omega} u \, \dive(\Psi) \diff x = - \int_{\Gamma_D} (u_D \Psi) \cdot n \diff a\\
		&\int_{\Omega} \dive(q) \, \Phi \diff x = 0
	\end{cases}	\\
	\text{ für alle } (\Psi, \Phi) \text{ mit } \Psi \cdot n = 0 \text{ auf } \Gamma_N 
\end{gather*}
\textbf{Idee der gemischten Finiten Elemente}: Diskretisieren wir (gFE), so erfüllt unsere numerische Lösung $ q_h $ die Forderung $ \dive(q_h) = 0 \text{ auf } K \ (K \in \mathcal{K}) $ und somit auch die Flussbilanz in jeder Zelle.

\subsection{Diskretisierung}

Sei $ \mathcal{K} $ eine zulässige Triangulierung von $ \Omega $ ($ \Omega = \bigcup_{K \in \mathcal{K}} K $).

Weiter sei $ \mathcal{F} $ die \defi{Menge aller Seiten}, $ \mathcal{F}_K $ die \defi{Menge aller Seiten von $ K $} und $ \partial \Omega_h $ der \defi{Rand von $ \Omega_h $}, formal definiert mit
\begin{align*}
	\mathcal{F} &\coloneqq \left( \{ \partial K_1 \cap \partial K_2 : K_1,K_2 \in \mathcal{K} \} \cup \{ \partial K_1 \cap \partial \Omega : K_1 \in \mathcal{K}\} \right) \setminus \{ \emptyset \} \\
	\mathcal{F}_K &\coloneqq \left( \{ \partial K \cap \partial K' : K' \in \mathcal{K} \} \cup \{ \partial K \cap \partial \Omega\} \right) \setminus \{ \emptyset \} \\
	\partial \Omega_h &\coloneqq \bigcup_{F \in \mathcal{F}} F.
\end{align*}
Wir nummerieren Seiten und Zellen durch:
\begin{align*}
	\mathcal{F} &= \{F_1, \dots , F_{\abs{\mathcal{F}}} \} &\text{\defi{globale Seitennummerierung}} \\
	\mathcal{K} &= \{K_1, \dots , K_{\abs{\mathcal{K}}} \} &\text{\defi{globale Zellennummerierung}} 
\end{align*}
\begin{remark}
	\begin{gather*}
	\text{Es gilt }\Omega = \Omega_h \cupdot \partial_h\Omega \cupdot \mathcal{V}_h. \\
	\Omega_h = \bigcup_{K \in \mathcal{K}} K \text{ (Flächen/Zellen)}\quad \partial_h\Omega = \bigcup_{F \in \mathcal{F}} F \text{ (Seiten)}\quad \mathcal{V}_h = \bigcup_{z \in \mathcal{V}}\{ z \} \text{ (Knoten)}	
	\end{gather*}
\end{remark}
\begin{define}
	Als Testräume/FE-Räume wählen wir $ W_h, Q_h $ und $ W_h(-g_N) $
	\begin{align*}
		W_h &\coloneqq \spann \left\{ \psi_1, \dots , \psi_{\abs{\mathcal{F}}}  \right\} &&\text{(\defi{Seitenansatzraum}/ Raum für $ \psi $ und $ q_h $ mit RWen)}\\
		Q_h &\coloneqq \spann \{ \mu_1, \dots , \mu_{\abs{\mathcal{K}}} \} &&\text{(\defi{Zellenansatzraum}/ Raum für $ \phi $ und $ u_h $)}
	\end{align*}
	\begin{align*}
		W_h(g) &\coloneqq \left\{ \psi_h \in W_h: \int_F \psi_h \cdot n \diff a = \int_F g \diff a \quad (F \subseteq \Gamma_N) \right\}.
	\end{align*}
	$ \{ \psi_i \}_{i=1}^{\abs{\mathcal{F}}} $ heißt \defi{Seitenbasis} und ist gegeben durch 
	\begin{align*}
		\forall i,j \in \{1, \dots , \abs{\mathcal{F}} \} : \int_{F_j} \psi_i \cdot n^K \diff a = \pm \delta_{i,j} \text{ und }  \psi_i|_K \in \mathbb{P}_1(K,\R^2) \cap C(\overline{\Omega}) \ (K \in \mathcal{K}) 
	\end{align*}
	und  $ \{ \mu_i \}_{m=1}^{\abs{\mathcal{K}}} $ heißt \defi{Zellenbasis} und ist gegeben durch
	\begin{align*}
		\forall m \in \{1, \dots , \abs{\mathcal{K}} \} : \mu_m \coloneqq  \mathbbm{1}_{K_m}.
	\end{align*}
\end{define}

\begin{remark}
	
	\[	\forall K \in \mathcal{K}:\psi_i|_K \in \mathbb{P}_1(K,\R^2) \text{ und } \mu_m|_K \in \mathbb{P}_0(K,\R) \]
	Also
	\begin{align*}	
	W_h &\subseteq \prod_{K \in \mathcal{K}} \mathbb{P}_1(K,\R^2) &&\text{(Menge der zellenweisen linearen Funktionen) und }\\
	Q_h &\subseteq \prod_{K \in \mathcal{K}} \mathbb{P}_0(K,\R) &&\text{(Menge der zellenweisen konstanten Funktionen)}. 
	\end{align*}	
\end{remark}

\begin{define}
	Weiter definieren wir das \defi{diskrete gemischte Finite-Elemente-Problem}:
	\begin{gather*}
		\text{Bestimme } (q_h,u_h) \in W_h(-g_N) \times Q_h \text{ mit}\\
		\dgFE\begin{cases}
			\int_{\Omega} \kappa ^{-1} q_h \cdot \psi_h \diff x \mkern-15mu &- \int_{\Omega} u_h \dive(\psi_h) \diff x = - \int_{\Gamma_D} u_D \psi_h \cdot n \diff a\\
			&- \int_{\Omega} \dive(q_h) \phi_h \diff x = 0
		\end{cases} \\
		\text{für alle } (\psi_h, \phi_h) \in W_h(0) \times Q_h
	\end{gather*}
\end{define}
\subsubsection{Referenzzelle}

Es bleibt die Frage offen, wie wir die Seitenbasis $ \{ \psi_i \}_{i=1}^{\abs{\mathcal{F}}} $ erhalten. Dazu betrachten wir, wie im Abschnitt zu Finiten Elementen, die Referenzzelle. Im Folgenden werden wir das Referenzdreieck anschauen, analog können die Schritte für das Referenzviereck vollzogen werden. 

\begin{define}
	
	Das \defi{Referenzdreieck} $ \triangle $ ist definiert als
	 \[ \hat{K} \coloneqq \conv \{ \hat{\mathcal{V}} \} \text{, wobei } \hat{\mathcal{V}} \coloneqq \left\{ 
	 \begin{pmatrix}
		 0\\
		 0
	 \end{pmatrix},
	 \begin{pmatrix}
		 1\\
		 0
	 \end{pmatrix},
	 \begin{pmatrix}
		 0\\
		 1
	 \end{pmatrix} \right\} \]
	Die Seiten des Referenzdreiecks sind
	\begin{align*}
	 	\hat{F}_0 &\coloneqq \conv \left\{ 
	 	\begin{pmatrix}
	 		0\\
	 		0
	 	\end{pmatrix},
	 	\begin{pmatrix}
		 	1\\
		 	0
	 	\end{pmatrix} \right\}\\
	 	\hat{F}_1 &\coloneqq \conv \left\{ 
	 	\begin{pmatrix}
		 	1\\
		 	0
	 	\end{pmatrix},
	 	\begin{pmatrix}
		 	0\\
		 	1
	 	\end{pmatrix} \right\}\\
	 	\hat{F}_2 &\coloneqq \conv \left\{ 
	 	\begin{pmatrix}
		 	0\\
		 	1
	 	\end{pmatrix},
	 	\begin{pmatrix}
		 	0\\
		 	0
	 	\end{pmatrix} \right\}\\ 	
	\end{align*}
	die Seitenbasis ist gegeben durch
	\begin{align*}
	 	\hat{\psi}_0:\hat{K} \to \R^2, \hat{\psi}_0(\xi) &\coloneqq 
	 	\begin{pmatrix}
	 		\xi_1\\
	 		\xi_2-1
	 	\end{pmatrix}\\
	 	\hat{\psi}_1:\hat{K} \to \R^2, \hat{\psi}_1(\xi) &\coloneqq  
	 	\begin{pmatrix}
		 	\xi_1\\
		 	\xi_2
	 	\end{pmatrix}\\
	 	\hat{\psi}_2:\hat{K} \to \R^2, \hat{\psi}_2(\xi) &\coloneqq 
	 	\begin{pmatrix}
		 	\xi_1-1\\
		 	\xi_2
	 	\end{pmatrix}.
	\end{align*}
	und $ \hat{n} $ sei der äußere Normalenvektor von $ \hat{K} $
\end{define}

\begin{remark}
 	$\forall i,j\in\{1,2,3\}: \int_{F_j} \hat{\psi}_i \cdot \hat{n} \diff a = \delta_{i,j} \text{ und } \hat{\psi}_i \in \mathbb{P}_1(\hat{K}, \R^2).$
\end{remark}

Weiter setzen wir noch 
\begin{align*}
\text{ die Menge der Seiten }&& \hat{\mathcal{F}} &\coloneqq \left\{ \hat{F}_0, \hat{F}_1, \hat{F}_2 \right\} \\
\text{und den Seitenansatzraum }&&  \hat{W} &\coloneqq \spann \left\{ \psi_0,\psi_1, \psi_2 \right\}.
\end{align*}
\textbf{Transformation von $ \hat{K} $ zu $ K $:} Für ein beliebiges $ K \in \mathcal{K} $ wollen wir jetzt eine Seitenbasis $ \{ \psi_1^K, \psi_2^K, \psi_3^K \} $ berechnen (Wie bisher gegeben durch $ \forall i\in\{1,2,3\}: \psi_i^K\in\mathbb{P}_1(K,\R^2) \text{ und } \int_{F_j^K} \psi_i^K \cdot n^K \diff a = \delta_{i,j} $, wobei $ n^K $ äußere Normale von  $ K $ und $ F_j^K  $ beliebige Seite von $ K $). Dazu betrachten wir die affine \defi{Transformationsabbildung $ \varphi_K $} von $ \hat{K} $ zu $ K $:
\begin{gather*}
	\varphi_K: \hat{K} \to K, \varphi_K(\xi) = z_{0,K} + B_K \xi \text{ mit passenden } B_K\in\R^{2 \times 2} \text{ und } \\
	J_K \coloneqq \det(B_K) > 0.
\end{gather*}
\begin{Lemma}
	Es gilt: $\tilde{n}^K = \frac{1}{\abs{B_K^{-T} \hat{n}}} B_K^{-T} \hat{n} $ ist Normale zu $ \partial K $.
\end{Lemma}
\begin{remark}
	\unklar{
	Damit $ \tilde{n}^K $ zu einer äußeren Normalen von $ K $ wird, betrachten wir für $ \hat{F} = \conv \{\hat{z}_j, \hat{z}_k\} $ als Seite von $ \hat{K} $ und $ F = \varphi_K(\hat{F}) = \conv \{ z_j , z_k \}$ als Seite von $ K = \varphi_K(\hat{K}) $:
	\begin{gather*}
		u_F \coloneqq\frac{1}{\abs{z_j-z_k}} 
		\begin{pmatrix}
			z_j[2] - z_k[2] \\
			z_k[1] - z_j[1]
		\end{pmatrix} \quad
		u_{\hat{F}} \coloneqq \frac{1}{\abs{\hat{z}_j-\hat{z}_k}} 
		\begin{pmatrix}
			\hat{z}_j[2] - \hat{z}_k[2] \\
			\hat{z}_k[1] - \hat{z}_j[1]
		\end{pmatrix} \\
		s_{K,F} \coloneqq \sign(u_F \cdot B_F^{-T} u_{\hat{F}}).
	\end{gather*} 
	Wir erhalten die äußere Normale durch 
	\[ n^K = s_{K,F} \; \tilde{n}^K \]
	}
\end{remark} 
Weiter bekommen wir unsere Seitenbasis auf $ K $ durch 
\[ \psi_i^K = J_K^{-1}B_K \hat{\psi}_i \circ \varphi_K \ (i\in \{1,2,3\}) \]
\begin{remark}
	\textcolor{red}{Es gilt: $ \hat{\psi}_i \cdot \hat{n} = 0 \iff \psi_i^K \cdot n^K = 0 $}
\end{remark}
Um  an unsere Seitenbasis auf $ \Omega \ \{\psi_j\}_{j = 1}^{\abs{\mathcal{F}}}$ zu kommen, benötigen wir noch die Übersetzung der globalen Seitennummerierung in einer Zelle $ K $ zur globalen Seitennummerierung
\[ l:\mathcal{K}\times \{1,2,3\} \to \{1, \dots , \abs{\mathcal{F}} \}, (K,i) \mapsto l(K,i). \]
Wir setzen $ \psi_j (j \in \{1,\dots , \abs{\mathcal{F}}\})$ durch
\[ \psi_j(x) = 
\begin{cases}
	\psi_i^K(x) , &\text{falls } j = l(K,i)\\
	0,			  &\text{sonst}.
\end{cases} \]
\begin{remark}
	Für alle Zellen $ K \in \mathcal{K} $ von denen $ F_j $ eine anliegende Seite ($ \overline{K} \cap F_j \ne \emptyset $) und $ F_j $ lokal mit $ i \in \{1,2,3\} $ nummeriert ist, gilt:
	\[ \psi_j|_K = \psi_i^K. \]
\end{remark}

\subsection{Formulierung als LGS}
Seien wie oben Seiten, Seitenbasis, Zellen und Zellenbasis global nummeriert 
\begin{align*}
&N \coloneqq \abs{\mathcal{F}} &&\mathcal{F} = \{F_1, \dots , F_{N} \}  \\
&&&W_h =\{\psi_1, \dots , \psi_N\}\\
&M \coloneqq \abs{\mathcal{K}} &&\mathcal{K} = \{K_1, \dots , K_{M} \}  \\
&&&Q_h = \{\mu_1 , \dots , \mu_M  \}.
\end{align*}

\begin{define}
	\begin{align*}
		&\underline{A} \in \R^{N \times N} \text{ mit } \underline{A}[n,k] \coloneqq \int_{\Omega} \kappa^{-1} \psi_n \cdot \psi_k \diff x \\
		&\underline{B} \in \R^{M \times N} \text{ mit } \underline{B}[m,k] \coloneqq - \int_{\Omega} \mu_m \dive(\psi_k) \diff x \\
		&\underline{b} \in \R^N \text{ mit } \underline{b}[k] \coloneqq - \int_{\Gamma_D} u_D \psi_k \cdot n \diff a
	\end{align*}
	und (für die Randbedingungen)
	\begin{gather*}
		\underline{W}(g) \coloneqq \left\{ \underline{q} \in \R^N : \underline{q}[k] = \int_{F_k} g  \diff a \ (\text{für } k \text{ mit } F_k \subseteq \Gamma_N) \right\} 
	\end{gather*}
\end{define}

\textbf{Umformung:} Unser zu lösendes Problem (dgFE) lässt sich mit $ q_h = \sum_{n=1}^{N} \underline{q}[n] \psi_n $ und $ u_h = \sum_{m=1}^{M} \underline{u}[m] \mu_m $ umformen zu 
\begin{gather*}
	\text{Bestimme} (\underline{q},\underline{u}) \in \underline{W}(-g_N)\times \R^M \text{ mit }\\
	\text{(L gFE)}\begin{cases}
		\underline{A} \underline{q} + \underline{B}^T \underline{u} &= \underline{b} \\
		\underline{B} \underline{q} &= 0
	\end{cases}
\end{gather*}
oder anders geschrieben 
\begin{gather*}
\text{Bestimme} (\underline{q},\underline{u}) \in \underline{W}(-g_N)\times \R^M \text{ mit }\\
	\text{(uL gFE)}\begin{cases}
		\begin{pmatrix}
			\underline{A} &\underline{B}^T\\
			\underline{B} &0
		\end{pmatrix}
		\begin{pmatrix}
			 \underline{q} \\
			 \underline{u} 
		\end{pmatrix}
		 =
		 \begin{pmatrix}
			 \underline{b}\\
			 0
		 \end{pmatrix}.
	\end{cases}
\end{gather*}

\subsection{Optimierungstheorie und Sattelpunktproblem}
\begin{Lemma}\label{Sattelpunkt} Sei $ (\underline{q},\underline{u}) \in \underline{W}(-g_N) \times \R^M $ und $ L $ gegeben durch 
\[L: \underline{W}(-g_N) \times \R^M \to \R, (\underline{x}, \underline{y}) \mapsto \frac{1}{2} \underline{x}^T \underline{A} \underline{x} - \underline{b}^T \underline{x} + \underline{y}^T \underline{B} \underline{x} \]
und es gilt: $ \forall (\underline{r}, \underline{v}) \in \underline{W}(-g_N) \times \R^M: L(\underline{q}, \underline{v}) \leq L(\underline{q}, \underline{u}) \leq L(\underline{r}, \underline{u}) $ (insb. ist $ (\underline{q},\underline{u}) $ ein Sattelpunkt von $ L $).

Dann löst $ (\underline{q}, \underline{u}) $ (L gFE) bzw. $ (q_h,u_h) $ (dgFE) (wobei $ q_h = \sum_{n=1}^N \underline{q}[n] \psi_n $ und $ u_h = \sum_{m=1}^M\underline{u}[m] \mu_m $).
\end{Lemma}

\textbf{Aus Optimierungstheorie:} 
\begin{gather*}
	(\underline{q}, \underline{u}) \text{ Sattelpunkt von } L  \text { wie in Lemma \ref{Sattelpunkt}}\\
	\iff \underline{q} \text{ minimiert } E:W_h(-g_N) \to \R , \underline{x} \mapsto \frac{1}{2} \underline{x}^T \underline{A} \underline{x} - \underline{b}^T \underline{x} \text{ unter der NB } \underline{B} \underline{q} = 0.
\end{gather*}

\begin{Satz}[Dual-Primal-Error] ~\newline
	Wir betrachten unser Ausgangsproblem \emph{(LG)} bzw. \emph{(LG F)}. Sei $ u_h \in V_h(u_D) $ Lösung des linearen FE-Verfahren (\defi{primale Lösung}) und $ q_h \in W_h(-g_N) $ Lösung des gemischten FE-Verfahren (\defi{duale Lösung}). 
	
	Dann gilt die A-posteriori Schranke (wobei  $ \norm{f}_{\kappa^{-1}} = \int \kappa^{-1} f \cdot f \diff x $ ) 
	\begin{gather*}
	\norm{q-q_h}_{\kappa^{-1}} \leq \underbrace{\norm{q_h + \kappa \nabla u_h}_{\kappa^{-1}}}_{\defi{Dual-Primal-Error}} 
	\end{gather*}
\end{Satz}


\subsection{Hybridisierung}

%Beim Lösen von (uL gFE) gibt es das Problem, dass die Matrix \[ \begin{pmatrix}
%\underline{A} &\underline{B}^T\\
%\underline{B} &0
%\end{pmatrix} \] im Allgemeinen nicht symmetrisch positiv definit ist und somit das CG-Verfahren nicht zum Lösen verwendet werden kann. Durch die Hybridisierung erhalten wir ein LGS mit einer spd Matrix.

Wir betrachten die Räume
	\[W_K \coloneqq \left\{ \psi_K:K \to \R^2: \psi_K = J_K^{-1}  B_K \hat{\psi} \circ \varphi_K^{-1}, \hat{\psi} \in \hat{W} \right\} \]
\[ W_\mathcal{K} \coloneqq \prod_{K \in \mathcal{K}} W_K, \qquad M_h \coloneqq \prod_{F \in \mathcal{F}} \mathbb{P}_0(F) \]
\[ M_h(u_D) \coloneqq \left\{ \mu_h \in M_h:\forall  F \subset \Gamma_D \int_F \mu_h \diff a = \int_F u_D \diff a  \right\}\]

\begin{remark}~
	
	$ \psi_h \in W_h \iff \big[ \psi_h \in W_{\mathcal{K}} \text{ und }  (\psi_{K_1} - \psi_{K_2}) \cdot n^F = 0 \ (F= \partial K_1 \cap \partial K_2 \in \mathcal{F}^{\circ})\big] $
\end{remark}


Und untersuchen folgende Problem:
\begin{gather*}
	\text{Bestimmme } (q_h,u_h, \lambda_h) \in W_\mathcal{K} \times Q_h \times M_h(u_D) \text{ mit}\\
	\hFgFE\begin{dcases}
		(1) \int_K \kappa^{-1} q_h \psi_K \diff x - \int_K u_h \dive(\psi_K) \diff x = \int_{\partial K} \lambda_h \psi_K \cdot n^K \diff a \\
		(2) \int_K \dive(q_h) \psi_K \diff x = 0\\
		(3) \sum_{K\in \mathcal{K}} \int_{\partial K} q_h \cdot n \mu_h \diff a = - \int_{\Gamma_N} g_N \mu_h \diff a
	\end{dcases}\\
	\text{für alle } K \in \mathcal{K}, \psi_K \in W_K  \text{ und } \mu_h \in M_h(0)
\end{gather*}

Dieses Problem ist äquivalent zu dem diskreten gemischten FE-Problem $ \dgFE $: (vgl. Vorlesung)
\begin{gather*}
\text{Bestimme } (q_h,u_h) \in W_h(-g_N) \times Q_h \text{ mit}\\
\dgFE\begin{cases}
\int_{\Omega} \kappa ^{-1} q_h \cdot \psi_h \diff x \mkern-15mu &- \int_{\Omega} u_h \dive(\psi_h) \diff x = - \int_{\Gamma_D} u_D \psi_h \cdot n \diff a\\
&- \int_{\Omega} \dive(q_h) \phi_h \diff x = 0
\end{cases} \\
\text{für alle } (\psi_h, \phi_h) \in W_h(0) \times Q_h
\end{gather*}

Für ein festes $ K \in \mathcal{K} $ ergibt sich mit der Wahl einer Basis von $ W_K $, $ Q_h $ und $ M_h $ 
%\begin{align*}
%	&W_K = \spann\{\psi_F \mid F \in \mathcal{F}_K \}  &&(\psi_F \cdot n^F)_{|F'} = \begin{cases}
%		1 &F = F'\\
%		0 &\text{sonst}
%	\end{cases}\\
%	&Q_h = \spann\{\eta_K \mid K \in \mathcal{K}  \}  &&{\eta_K}_{|K'} = \begin{cases}
%	1 &, K = K'\\
%	0 &,\text{sonst}
%	\end{cases}\\
%	&M_h = \spann \{ \nu_F \mid F \in \mathcal{F}_K \}  &&{\nu_F}_{|F'} = \begin{cases}
%		1 &, F = F'\\
%		0 &, \text{sonst}
%	\end{cases}  
%\end{align*}
 
eine Formulierung als LGS mit Nebenbedingung, wobei $ \ubar{q}_K \coloneqq \ubar{R}_K \ubar{q}, \ \ubar{u}_K \coloneqq \ubar{R}_K \ubar{u} $
\begin{gather*}
	\text{Bestimme } \ubar{q}, \ubar{u} \text{ und } \ubar{\lambda} \text{ mit}\\
	\begin{dcases}
		(1) \begin{pmatrix}
			\ubar{A}_K & \ubar{B}_K \\
			\ubar{B}_K^T & 0
		\end{pmatrix}
		\begin{pmatrix}
			\ubar{q}_K\\
			\ubar{u}_K
		\end{pmatrix}
		\mkern-16mu&= \begin{pmatrix}
			- \ubar{C}_K \, \ubar{R}_K \ubar{\lambda} \\
			0
		\end{pmatrix}\\
		(2) \sum_{K \in \mathcal{K}}  (\ubar{R}_K \ubar{\mu})^T \ubar{C}_K \ubar{q}_K \mkern-16mu&= \ubar{\mu}^T \ubar{b}		
	\end{dcases}\\
	\text{für alle } \ubar{\mu} \text{ mit } \ubar{\mu}[F] = 0 \text{ für } F \in \Gamma_D \cap \mathcal{F}
\end{gather*}

\begin{gather*}
	\left(
		\begin{array}{c}
		1\\ 1 \\ \vdots \\ 1 \\ 1\\ \ubar{\mu}
		\end{array}
	\right)^T
	\underbrace{\left(
	\begin{array}{ccccc|c}
	\ubar{A}_{K_1} & \ubar{B}_{K_1}&&&& \ubar{C}_{K_1} \ubar{R}_{K_1}\\
	\ubar{B}_{K_1} & 0 &&&& 0\\
	&& \ubar{A}_{K_2} & \ubar{B}_{K_2} && \ubar{C}_{K_2} \ubar{R}_{K_2}\\
	 &  & \ubar{B}_{K_2} & 0 && 0\\
	 &  &  &  &\ddots &\\
	\hline
	\ubar{R}_{K_1}^T \ubar{C}_{K_1}^T& 0 & \ubar{R}_{K_2}^T \ubar{C}_{K_2}^T& 0& &0 \\
	\end{array}
	\right)}_{\eqqcolon \left( \begin{array}{c|c}
	\ubar{D} & \ubar{E}\\
	\hline
	\ubar{E}^T & 0
	\end{array} \right)}
	\underbrace{\left(
	\begin{array}{c}
		\ubar{q}_{K_1}\\ \ubar{u}_{K_1} \\ \ubar{q}_{K_2}\\ \ubar{u}_{K_2} \\ \vdots \\ \ubar{\lambda}
	\end{array}
	\right)}_{\eqqcolon \left( \begin{array}{c}
		\left(
		\begin{array}{c}
			\ubar{q}_{K_i}\\ \ubar{u}_{K_i}
		\end{array}
		\right)_{K_i \in \mathcal{K}} \\ \ubar{\lambda} 
	\end{array}\right)} = 
	\left(
	\begin{array}{c}
	0\\ 0 \\ \vdots \\ 0 \\ 0\\ \ubar{\mu}^T \ubar{b}
	\end{array}
	\right)
\end{gather*}
Mit dem Schurkomplement $ \ubar{S} \coloneqq \ubar{E}^T \ubar{D}^{-1} \ubar{E} $ folgt
\[ \ubar{\mu}^T \ubar{S} \, \ubar{\lambda} = \ubar{\mu}^T \ubar{b} 	\text{ für alle } \ubar{\mu} \text{ mit } \ubar{\mu}[F] = 0 \text{ für } F \in \Gamma_D \cap \mathcal{F} \]

Sobald wir $ \ubar{\lambda}_k \coloneqq \ubar{R}_K \ubar{\lambda}$ bestimmt haben, können wir auch das obere LGS (1) lösen um $ \ubar{q}_K $ und $ \ubar{u}_K $ zu erhalten.
% Um $ \ubar{\lambda}_k $ zu erhalten lösen wir das globales System:
%\begin{gather*}
%	\underline{S} \, \underline{\lambda} = \underline{b}
%\end{gather*}
%und bekommen somit $ \ubar{\lambda}_k $ über die Restriktionen
%\[ \ubar{\lambda} = \sum_{K \in \mathcal{K}} \ubar{R}_K \ubar{\lambda}_K .\]



