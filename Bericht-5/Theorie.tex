% !TeX root = Bericht_main.tex

\underline{\textbf{Ziel:}} Übertragung der dG-Diskretisierung für die lineare Transportgleichung $ \lTG $ (hyperbolische) auf elliptische Gleichungen (z.B. Potentialströmungen).

Wir betrachten erneut das Potentialströmungsproblem
\begin{gather*}
	\text{Bestimme } (q, u) \text{, sodass}\\
	\PS \begin{cases}
		\dive(q) = 0 &\text{, in }\Omega\\
		q = - \kappa \nabla u &\text{, in } \Omega\\
		-q \cdot \nu = g_N &\text{, auf }\GammaN\\
		u = u_D &\text{, aud } \GammaD. 
	\end{cases}
\end{gather*}
Dabei ist $ \partial \Omega = \GammaN \cupdot \GammaD $. Sei weiter $ \mathcal{K} $ eine zulässige Triangulierung mit Kantenmenge $ \mathcal{F} $ und Eckenmenge $ \mathcal{V} $. Außerdem sei $ \Omega_h \coloneqq \bigcup_{K \in \mathcal{K}} K$ und $ \partial \Omega_h \coloneqq \overline{\Omega} \setminus \Omega_h $.

Es gilt
\begin{gather*}
	\oint_{\partial K} \phi_h q \cdot n^K \diff a = \int_K \dive(\phi_h q) \diff x = \int_K \nabla\phi_h \cdot q \diff x + \int_K \phi_h \underbrace{\dive(q)}_{=0} \diff x\\
	\implies \oint_{\partial K} \phi_h q \cdot n^K \diff a = \int_K \nabla\phi_h \cdot q \diff x \ (K\in\mathcal{K})
\end{gather*}



