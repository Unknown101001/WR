% !TeX root = Bericht_main.tex

\underline{\textbf{Ziel:}} Übertragung der dG-Diskretisierung für die lineare Transportgleichung $ \lTG $ (hyperbolische) auf elliptische Gleichungen (z.B. Potentialströmungen).

Wir betrachten erneut das Potentialströmungsproblem
\begin{gather*}
	\text{Bestimme } (q, u) \text{, sodass}\\
	\PS \begin{cases}
		\dive(q) = 0 &\text{, in }\Omega\\
		q = - \kappa \nabla u &\text{, in } \Omega\\
		q \cdot n = g_N &\text{, auf }\GammaN\\
		u = u_D &\text{, aud } \GammaD. 
	\end{cases}
\end{gather*}
Dabei ist $ \partial \Omega = \GammaN \cupdot \GammaD $. Sei weiter $ \mathcal{K} $ eine zulässige Triangulierung mit Kantenmenge $ \mathcal{F} $ und Eckenmenge $ \mathcal{V} $. Außerdem sei $ \Omega_h \coloneqq \bigcup_{K \in \mathcal{K}} K$ und $ \partial \Omega_h \coloneqq \overline{\Omega} \setminus \Omega_h $.

Es gilt 
\begin{align*}
	&\forall K \in \mathcal{K}: \oint_{\partial K} \phi_h q \cdot n^K \diff a = \int_K \dive(\phi_h q) \diff x = \int_K \nabla\phi_h \cdot q \diff x + \int_K \phi_h \underbrace{\dive(q)}_{=0} \diff x\\
	\implies &\forall K \in \mathcal{K}: -\oint_{\partial K} \phi_h q \cdot n^K \diff a + \int_K \nabla\phi_h \cdot q \diff x  = 0 \\
	\implies &\sum_{K \in \mathcal{K}} \left( - \int_K q \cdot \nabla \phi_h \diff + \oint_{\partial K \setminus \GammaN} q \cdot n \underbrace{\phi_K}_{\coloneqq \phi_{h|K}} \diff a  \right) = \oint_{\GammaN} g_N \phi_h \diff a
\end{align*}

\begin{Satz}(\defi{DG-Formel})\\
	Sei $ \overline{F} = \partial K \cap \partial K_F $ mit $ K \ne K_F $ (also ist $ F $ eine innere Kante) und $ ( \underbrace{q_{K} }_{\coloneqq q_{|K}}- \underbrace{q_{K_F}}_{\coloneqq q_{|K_F}} )  \cdot n^K = 0$.
	
	Dann gilt:
	\[ q_K \cdot n^K \phi_K + q_{K_F} \cdot n^{K_F} \phi_{K_F} = \{q\}_F \cdot [\phi_h]_F \]
	mit
	\begin{align*}
		\{q\}_F &\coloneqq \frac{1}{2} (q_K + q_{K_F})\\
		[\phi_h]_F &\coloneqq \underbrace{\phi_K \, n^K + \phi_{K_F} \, n^{K_F}}_{\substack{\text{Benötigt keine}\\ \text{Vorzeichen-/Richtungswahl}}} = (\phi_K - \phi_{K_F}) \, n^K
	\end{align*}
\end{Satz}

Wir wollen wieder Testfunktionen $ \phi_h \in Q_h \coloneqq \Pi_{K \in \mathcal{K}} \mathbb{P}_k(K)$ betrachten, wie beim letzten Mal bei \enquote{dG-Verfahren}. Da diese Funktionen sind nur auf $ K \in \mathcal{K} $, aber nicht unbedingt auf $ \Omega $. Es gilt
\begin{align*}
	&\oint_{\partial K} \kappa \phi_h \nabla u \cdot n \diff x = \int_K \dive(\underbrace{\kappa \nabla u}_{= -q} \phi_h) \diff x = \int_K \underbrace{\dive(-q)}_{=0} \phi_h + (-q) \nabla \phi_h \diff x = \int_K \kappa \nabla u \cdot \nabla \phi_h \diff x\\
	\implies &\int_{\GammaN} g_N \phi_h \cdot n \diff a = \int_K \kappa \nabla u \cdot \nabla \phi_h \diff x - \int_{\partial K \setminus \GammaN} \kappa \nabla u \phi_h \cdot n \diff a\\
	\implies &\int_{\GammaN} g_N \phi_h \cdot n \diff a = \int_K \kappa \nabla u \cdot \nabla \phi_h \diff x + \int_{\partial K \setminus \GammaN} q \phi_h \cdot n \diff a\\
	%\implies &\sum_{K \in \mathcal{K}} \int_K \kappa \nabla u \cdot \nabla \phi_h \diff x - \sum_{F \in \mathcal{F} \cap \Omega_h} \int_{F} \{\kappa \nabla u \}_F [\phi_h]_F \diff a - \sum_{F \in \mathcal{F} \cap \GammaD} \int_{F} \kappa \nabla u \phi_h \cdot n \diff a = \int_{\GammaN} g_N \phi_h \cdot n \diff a\\
	\implies &\sum_{K \in \mathcal{K}} \int_K \kappa \nabla u \cdot \nabla \phi_h \diff x - \sum_{F \in \mathcal{F} \setminus \GammaN} \int_{F} \{\kappa \nabla u \}_F [\phi_h]_F \diff a = \int_{\GammaN} g_N \phi_h \cdot n \diff a
\end{align*}


