% !TeX root = Bericht_main.tex

\underline{\textbf{Ziel:}} Übertragung der dG-Diskretisierung für die lineare Transportgleichung $ \lTG $ (hyperbolische) auf elliptische Gleichungen (z.B. Potentialströmungen).

Wir betrachten erneut das Potentialströmungsproblem
\begin{gather*}
	\text{Bestimme } (q, u) \text{, sodass}\\
	\PS \begin{cases}
		\dive(q) = 0 &\text{, in }\Omega\\
		q = - \kappa \nabla u &\text{, in } \Omega\\
		-q \cdot n = g_N &\text{, auf }\GammaN  \\
		u = u_D &\text{, auf } \GammaD. 
	\end{cases}
\end{gather*}
Dabei ist $ \partial \Omega = \GammaN \cupdot \GammaD $. Sei weiter $ \mathcal{K} $ eine zulässige Triangulierung mit Kantenmenge $ \mathcal{F}_h $ und Eckenmenge $ \mathcal{V} $. Außerdem sei $ \Omega_h \coloneqq \bigcup_{K \in \mathcal{K}} K$ und $ \partial \Omega_h \coloneqq \overline{\Omega} \setminus \Omega_h $.

Weiter wollen wir wieder Testfunktionen $ \phi_h \in Q_h \coloneqq \Pi_{K \in \mathcal{K}} \mathbb{P}_k(K)$ betrachten, wie beim letzten Mal bei \enquote{dG-Verfahren}. Diese Funktionen sind nur stetig auf $ K \in \mathcal{K} $, aber nicht unbedingt auf $ \Omega $. 

%Es gilt 
%\begin{align*}
%	&\forall K \in \mathcal{K}: \oint_{\partial K} \phi_h q \cdot n^K \diff a = \int_K \dive(\phi_h q) \diff x = \int_K \nabla\phi_h \cdot q \diff x + \int_K \phi_h \underbrace{\dive(q)}_{=0} \diff x\\
%	\implies &\forall K \in \mathcal{K}: -\oint_{\partial K} \phi_h q \cdot n^K \diff a + \int_K \nabla\phi_h \cdot q \diff x  = 0 \\
%	\implies &\sum_{K \in \mathcal{K}} \left( - \int_K q \cdot \nabla \phi_h \diff + \oint_{\partial K \setminus \GammaN} q \cdot n \underbrace{\phi_K}_{\coloneqq \phi_{h|K}} \diff a  \right) = \oint_{\GammaN} g_N \phi_h \diff a
%\end{align*}

\begin{Satz}(\defi{DG-Formel})\\
	Sei $ \overline{F} = \partial K \cap \partial K_F $ mit $ K \ne K_F $ (also ist $ F $ eine innere Kante) und $ ( \underbrace{q_{K} }_{\coloneqq q_{|K}}- \underbrace{q_{K_F}}_{\coloneqq q_{|K_F}} )  \cdot n^K = 0$.
	
	Dann gilt:
	\[ q_K \cdot n^K \phi_K + q_{K_F} \cdot n^{K_F} \phi_{K_F} = \{q\}_F \cdot [\phi_h]_F \]
	mit
	\begin{align*}
		\{q\}_F &\coloneqq \frac{1}{2} (q_K + q_{K_F})\\
		[\phi_h]_F &\coloneqq \underbrace{\phi_K \, n^K + \phi_{K_F} \, n^{K_F}}_{\substack{\text{Benötigt keine}\\ \text{Vorzeichen-/Richtungswahl}}} = (\phi_K - \phi_{K_F}) \, n^K
	\end{align*}
\end{Satz}

\begin{notation}~
	\begin{itemize}
		\item $ \FIn: $ Menge der inneren Kanten
		\item $ \FD: $ Menge der Kanten auf dem Dirichlet-Rand
		\item $ \FN: $ Menge der Kanten auf dem Neumann-Rand
	\end{itemize}
\end{notation}

Es gilt
\begin{align*}
	&\int_{\partial K} \kappa \phi_h \nabla u \cdot n \diff a = \int_K \dive(\underbrace{\kappa \nabla u}_{= -q} \phi_h) \diff x = \int_K \underbrace{\dive(-q)}_{=0} \phi_h + (-q) \nabla \phi_h \diff x = \int_K \kappa \nabla u \cdot \nabla \phi_h \diff x\\
	\implies &\int_{\GammaN} g_N \phi_h \diff a = \int_K \kappa \nabla u \cdot \nabla \phi_h \diff x - \int_{\partial K \setminus \GammaN} \kappa \nabla u \phi_h \cdot n \diff a\\
	%\implies &\int_{\GammaN} g_N \phi_h \diff a = \int_K \kappa \nabla u \cdot \nabla \phi_h \diff x + \int_{\partial K \setminus \GammaN} q \phi_h \cdot n \diff a\\
	\implies &\underbrace{\sum_{K \in \mathcal{K}} \int_K \kappa \nabla u \cdot \nabla \phi_h \diff x - \sum_{F \in \FIn} \int_{F} \{\kappa \nabla u \}_F [\phi_h]_F \diff a - \sum_{F \in \FD} \int_{F} \kappa \nabla u \phi_h \cdot n \diff a}_{\coloneqq a^{(1)}_h(u,\phi_h)}\\
	&\qquad = \underbrace{\int_{\GammaN} g_N \phi_h \diff a}_{\coloneqq b^{(1)}_h(\phi_h)}\\
	%&\text{Setzen wir nun für } F \subseteq \GammaD: \{\kappa \nabla u \}_F \coloneqq \kappa \nabla u \text{ und } [\phi_h]_F \coloneqq \phi_h \cdot n \text{ erhalten wir:}\\ 
	%\implies &\sum_{K \in \mathcal{K}} \int_K \kappa \nabla u \cdot \nabla \phi_h \diff x - \sum_{F \in \mathcal{F} \setminus \GammaN} \int_{F} \{\kappa \nabla u \}_F [\phi_h]_F \diff a = \int_{\GammaN} g_N \phi_h \cdot n \diff a
\end{align*}
Wir haben somit eine erste schwache Formulierung gefunden:
\begin{gather*}
\text{Bestimme } u_h \in Q_h \text{, sodass}\\
\begin{cases}
	a^{(1)}(u_h,\phi_h) = b^{(1)}(\phi_h) (\phi_h \in Q_h)\\
	u_h = u_D \text{ auf }\GammaD
\end{cases}
\end{gather*}

\begin{notation}
	Um die Notation übersichtlicher zu machen erweitern wir die Definition von $ \{\cdot\}_F $ und $ [\cdot]_F $. Sei $ h \in Q_h $ und $ F \in \mathcal{F}_h \setminus \GammaN $
	\begin{align*}
		&\{ h \}_F \coloneqq
		 \begin{cases}
					\frac{1}{2} (h_K + h_{K_F} ) &\text{für } F \subseteq \Omega\\
					g &\text{für } F \subseteq \GammaD
		\end{cases}
	  &[h]_F \coloneqq \begin{cases}
			h_K n^K + h_{K_F} n^{K_F} &\text{für } F \subseteq \Omega\\
			h \, n &\text{für } F \subseteq \GammaD
		\end{cases}
	\end{align*}
	
	Somit lässt sich $ a^{(1)} $ schreiben als
	\[a^{(1)}(u_h, \phi_h) = \sum_{K \in \mathcal{K}} \int_K \kappa \nabla u \cdot \nabla \phi_h \diff x - \sum_{F \in \F \setminus \FN} \int_{F} \{\kappa \nabla u \}_F [\phi_h]_F \diff a. \]
\end{notation}





Wir wollen diese Formulierung des Problems allerdings noch ein wenig abändern. Zum einen fehlen noch die Dirichlet-RW in der Formulierung, zum anderen wünschen wir uns eine symmetrische, koerzive Bilinearform, damit unser numerisches Verfahren schöne Eigenschaften hat (z.B. die numerische Lösung bleibt beschränkt).\\
Im nächsten Schritt bauen wir die Dirichlet-RW in die schwache Formulierung ein:

Falls $ u_h = u_D $ auf $ \GammaD $, so gilt für $ F \subseteq \GammaD  $:
\begin{align}
	\label{D-RW1}& 0=\int_F (u_h - u_D) \phi_h \diff a &&\iff \int_F \underbrace{[u_h]_F [\phi_h]_F}_{= u_h \phi_h n \cdot n = u_h \phi_h} \diff a = \int_F u_D \phi_h \diff a  \\
	\label{D-RW2}& 0  = \int_F (u_h - u_D) \kappa \nabla \phi_h \cdot n \diff a  &&\iff \int_F [u_h]_F \cdot \{\kappa \nabla\phi_h\}_F \diff a = \int_F u_D \kappa \nabla \phi_h \cdot n \diff a 
\end{align} 
Wir wählen diese Formulierung, um die Forderung $ u_h = u_D $ auf $ \GammaD $ zu erfüllen. 
Außerdem können wir nun durch die Hinzunahme von weiteren Termen, welche für eine stetige Lösung $ u $ von $ \PS $ gleich 0 sind (sozusagen Strafterme für unstetige Lösungen), eine symmetrische koerzive Bilinearform erhalten.\\
Für eine Funktion $ u_h \in Q_h $ und $ \overline{F} = \partial K \cap \partial K_F $ mit $ u_K = u_{K_F} $ gilt $ [u_h]_F = 0 $ und somit
\begin{align}
	\label{innen1} &\int_F [u_h]_F \cdot [\phi_h]_F \diff a = 0\\
	\label{innen2}&\int_F [u_h]_F \cdot \{ \kappa\nabla\phi_h \}_F \diff a = 0
\end{align}

\begin{define}
	\begin{align*}
		&\bullet &\asip(u_h, \phi_h) \coloneqq &\underbrace{\sum_{K \in \mathcal{K}} \int_K \kappa \nabla u_h \cdot \nabla \phi_h \diff x - \sum_{F \in \F\setminus\FN } \int_{F} \{\kappa \nabla u_h \}_F [\phi_h]_F \diff a}_{= a_h^{(1)}(u_h, \phi_h)}\\
		&&&- \sum_{F \in \FIn} \sigma \underbrace{ \int_F [u_h]_F \cdot \{ \kappa \nabla \phi_h \}_F \diff a}_{(\ref{innen2})} - \sum_{F \in \FD} \sigma \underbrace{ \int_F [u_h]_F \cdot \{ \kappa \nabla \phi_h \}_F \diff a}_{(\ref{D-RW2})}\\
		&&&+\sum_{F \in \FIn} \gamma_F \underbrace{\int_F [u_h]_F \cdot [\phi_h]_F \diff a}_{(\ref{innen1})} + \sum_{F \in \FD} \gamma_F \underbrace{\int_F [u_h]_F \cdot [\phi_h]_F \diff a}_{(\ref{D-RW1})}\\
		&&=& \sum_{K \in \mathcal{K}} \int_K \kappa \nabla u_h \cdot \nabla \phi_h \diff x + \sum_{F \in \F\setminus\FN}  \gamma_F \int_F [u_h]_F \cdot [\phi_h]_F \diff a\\
		&&&- \sum_{F \in \F\setminus\FN} \int_F \left(  \{\kappa \nabla u_h \}_F [\phi_h]_F + \sigma [u_h]_F \{\kappa \nabla \phi_h \}_F  \right) \diff a\\
		&\bullet  &\bsip(\phi_h) \coloneqq & \underbrace{\int_{\GammaN} g_N \phi_h \diff a}_{=b_h^{(1)}(\phi_h)} \underbrace{ - \sigma \int_{\GammaD} u_D \kappa \nabla \phi_h \cdot n \diff a}_{(\ref{D-RW2})} + \underbrace{\sum_{F \in \FD} \gamma_F \int_F u_D \phi_h \diff a}_{(\ref{D-RW1})} \\
		&\bullet  &\norm{\phi_h}_{DG} \coloneqq  &\frac{1}{2} \left( \sum_{K \in \mathcal{K}} \int_{K} \kappa\nabla \phi_h \cdot \nabla \phi_h \diff x +  \sum_{F \in \F\setminus\FN }\gamma_F \int_F [ \phi_h ]_F^2 \diff a \right)
	\end{align*}
	wobei $ \sigma \in \{\pm 1\} $ und $ \gamma_F \ge 0 \ (F\in\F\setminus\FN)$.
	
	Diese Vorfaktoren werden in den Aufgaben verwendet: $ \sigma = 1 $ für \enquote{symmetric} bzw. $ \sigma = -1 $ für \enquote{non-symmetric} und $ \gamma_F $ als \enquote{penalty}.
	
Für $ \GammaD \ne \emptyset $ ist $ \norm{\cdot}_{DG} $ eine Norm auf $ Q_h(0) $.
\end{define}
%\begin{remark}\textcolor{red}{Bei Korrektur}\\
%	Das \enquote{$-$}-Zeichen vor dem zweiten Integral bei $ b_h $ sollte dort stehen, allerdings ist es nicht auf den Folien nicht so.
%\end{remark}

\paragraph{Eigenschaften:}
\begin{itemize}
	\item Für $ \sigma = 1 $ ist $ \asip(\cdot,\cdot) $ eine \emph{symmetrischen} Bilinearform und unter gewissen Voraussetzungen an $ \gamma_F $ ist $ \asip(\cdot,\cdot) $ \emph{koerzive} bezüglich der Norm $ \norm{\cdot}_{\text{DG}} $ (siehe VL. Lemma (7.2))
	\item $ \bsip $ ist ein beschränktes lineares Funktional bezüglich $ \norm{\cdot}_{\text{DG}} $ 
\end{itemize}

\paragraph{Konsistenz:}~\\
Für eine Lösung $ u $ von $ \PS $ mit $ u \in C(\overline{\Omega}) $ (u stetig auf $ \overline{\Omega} $) und   $q \cdot n|_K = q \cdot n|_{K_F} $ ($\rightarrow$ $q = - \kappa \nabla u$)
gilt die Gleichung
\[ \asip(u,\phi_h) = \bsip(\phi_h) \ (\phi_h \in Q_h \text{ mit } \phi_h = 0 \text{ auf } \GammaD) \]

\paragraph{Stabilität:}~\\
Sei $ u_h \in Q_h $ eine Lösung von $ \asip(u_h, \phi_h) = \bsip(\phi_h) \ (\phi_h \in Q_h) $ und $ \asip(u_h, \phi_h)$ eine koerzive Bilinearform. Dann gilt
\[ \norm{u_h}_{\text{DG}}^2 \le \asip(u_h, u_h) = \bsip(u_h) \le C \norm{u_h}_{\text{DG}} \implies \norm{u_h}_{\text{DG}} \le C \]
Also ist unsere numerische Lösung beschränkt. 

\paragraph{Wohldefiniertheit des Verfahrens:}~\\
Man kann zeigen, dass das durch $ \asip(u_h, \phi_h) = \bsip(\phi_h) \ (\phi_h \in Q_h) $ gegebene LGS eindeutig lösbar ist (da die durch $ \asip(\cdot, \cdot)  $ gegebene Abbildungsmatrix symetrisch positiv definit ist).